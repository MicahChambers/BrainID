\chapter{Conclusions}
\label{sec:Conclusion}
This work has demonstrated the use of the particle filter to
learn parameters of the BOLD model. Since the inception of the
BOLD equations, many attempts have been made use FMRI data to
to learn these parameters. These attempts have typically either been extremely
slow or relied on extensive assumptions. While the particle filter method
is not quick, 40 seconds to analyze each voxel is well within the capabilities
of the typical research lab. Previous attempts have also treated
the problem as if there were a single solution.

One significant finding of this work, that would not be clear without
calculating a full posterior distribution, is the interplay
between the parameters. The results of simulations clearly demonstrate
that identifying a single set of parameters is not possible with this
model. Although sensitivity tests in Deneux et al. certainly hinted
at this, the current work clearly demonstrates this fact \cite{Deneux2006}. Therefore,
any single estimate of parameters is insufficient for analysis. As such,
identification of the BOLD model that do not treat the parameters as distributions
will not be able to overcome the inherent ill-posed nature of the
parameters. Besides the Unscented Kalman Filter, this is the only approach
that accomplishes this task without repeatedly calculating the parameters
to build a distribution. The Unscented Kalman Filter though, is restricted
to Gaussian estimation which limits its ability to estimate the posterior.

The primary reason this method has not been used before is the high
dimensionality of the system. In Murray, 2008 the idea of learning
the parameters is floated as the best method, yet
learning 7 parameters with a Monte Carlo method was deemed intractable
\cite{Murray2008}. The
concern was that so many parameters creates a prohibitively large
search space, which requires too many particle to learn properly. To overcome
this difficulty, instead of starting the algorithm with 1000 particles,
the initial number of particles was set to 16,000. This meant that
when the search space was the largest, the number of particles was
sufficiently dense to represent the prior distribution. The result is
a particle filter algorithm that spends computing resources only where
it is really needed.

The particle filter is a Bayesian non-parametric algorithm that, given
enough measurements will approach the true probability distribution
of the parameters. The conclusion that the parameters are not
uniquely identifiable is disappointing in light of the plethora
of research papers that have attempted to learn them. However, the fact
that the parameters are ill-posed further demonstrates
the need to estimate a probability distribution rather than a single
parameter estimate. At the same time, the output of the particle filter
is still able to give good estimates of the BOLD output, and is not
dependent on heavy Gaussian smoothing. Therefore, all the current
experimental paradigms to determine regional activity are still viable.
To borrow a computer term, the particle filter algorithm is backward
compatible with the current methods. Indeed, the particle filter excelled
at determining activation in very low SNR simulations
(\autoref{sec:Multi-voxel Simulation}). Not only that, but the particle
filter algorithm identified areas of activation that were completely
missed by SPM (\autoref{fig:comp6}).

The physiological plausibility is also of great value to future
research. Unlike traditional methods, use of the BOLD model allows
the incorporation of outside, physiological knowledge in regression.
If, for instance, it becomes possible to measure $v$ in vivo that data
could instantly be added to the model to improve regression. Data can
also be retrieved from the model. BOLD parameter distributions could
for the first time be used as diagnosis tools. In the past the fact
that the parameters are
ill-posed would have prevented such research. However, now that a full
distribution is available, comparing parameter estimates is actually
feasible.

Concluding, the particle filter provides comparable performance with
conventional tests, but also provides a platform for a wide range of
future uses beyond what was possible with the methods that have been
used before.
