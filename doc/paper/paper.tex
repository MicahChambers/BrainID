%Journal Alticle Layout by Micah Chambers
%using the IEE bare_jrnl.tex file at 
%% http://www.michaelshell.org/tex/ieeetran/
%% http://www.ctan.org/tex-archive/macros/latex/contrib/IEEEtran/
%% and
%% http://www.ieee.org/

\documentclass[journal]{./IEEEtran}

\usepackage{cite}

\usepackage[pdftex]{graphicx}
\graphicspath{{pdf/}{png/}}

\usepackage[cmex10]{amsmath}
\interdisplaylinepenalty=2500

\usepackage{algorithmic}
\usepackage{array}

\usepackage[caption=false,font=footnotesize]{subfig}
\usepackage{fixltx2e}

\usepackage[colorlinks=true,linkcolor=black,citecolor=black]{hyperref}
\usepackage{url}
\hyphenation{op-tical net-works semi-conduc-tor}

%\begin{comment}
%:Author: Micah Chambers
%:Institution: Virginia Tech University
%\end{comment}

%\newcommand{\gettitle}{\text{Full Brain Blood-Oxygen-Level-Dependent Signal Parameter Estimation Using Particle Filters}}

\begin{document}

\title{Full Brain Blood-Oxygen-Level-Dependent Signal Parameter Estimation Using Particle Filters}

\author{{Micah~Chambers}% <-this % stops a space
\thanks{M. Chambers is with the Department
of Electrical and Computer Engineering, Virginia Tech University, Blacksburg,
VA, 24060 USA e-mail: (micahc@vt.edu}% <-this % stops a space
\thanks{Manuscript received April 19, 2005; revised January 11, 2007.}}

\markboth{Journal of \LaTeX\ Class Files,~Vol.~6, No.~1, January~2007}%
{Chambers: Full Brain Blood-Oxygen-Level-Dependent Signal Parameter Estimation Using Particle Filters}
% The only time the second header will appear is for the odd numbered pages
% after the title page when using the twoside option.
% 
% *** Note that you probably will NOT want to include the author's ***
% *** name in the headers of peer review papers.                   ***
% You can use \ifCLASSOPTIONpeerreview for conditional compilation here if
% you desire.

\maketitle

\begin{abstract}
Traditional methods of analyzing FMRI images use a linear combination of
just a few static regressors. This works demonstrates an alternative
approach using a physiologically inspired nonlinear model. By using a 
particle filter to optimize the model parameters, the computation time
is kept below a minute per voxel without requiring a linearization 
of the noise in the state
variables. The activation results show regions similar those found in 
SPM; however, there are some notable regions not detected by 
SPM. Though the parameters selected by the particle filter based approach
are more than sufficient to predict the BOLD response,
more model constraints are needed to uniquely identify a single set
of parameters. This ill-posed nature explains the large discrepancies
found in other research that attempted to characterize the model parameters.
For this reason the final distribution of parameters is more medically relevant
than a single estimate. Because the output of the particle filter is 
a full posterior probability, the reliance on the mean to estimate 
parameters is unnecessary. This work presents
not just a viable alternative to the traditional method of detecting
activation, but an extensible technique of estimating the joint Probability
Distribution Function of the BOLD parameters.
\end{abstract}

% Note that keywords are not normally used for peerreview papers.
\begin{IEEEkeywords}
BOLD Response, FMRI, Nonlinear Systems, Particle Filter, Bayesian Statistics, System Identification
\end{IEEEkeywords}

% For peer review papers, you can put extra information on the cover
% page as needed:
% \ifCLASSOPTIONpeerreview
% \begin{center} \bfseries EDICS Category: 3-BBND \end{center}
% \fi
%
% For peerreview papers, this IEEEtran command inserts a page break and
% creates the second title. It will be ignored for other modes.
\IEEEpeerreviewmaketitle


\section{Introduction}
% The very first letter is a 2 line initial drop letter followed
% by the rest of the first word in caps.
% 
% form to use if the first word consists of a single letter:
% \IEEEPARstart{A}{demo} file is ....
% 
% form to use if you need the single drop letter followed by
% normal text (unknown if ever used by IEEE):
% \IEEEPARstart{A}{}demo file is ....
% 
% Some journals put the first two words in caps:
% \IEEEPARstart{T}{his demo} file is ....
% 
% Here we have the typical use of a "T" for an initial drop letter
% and "HIS" in caps to complete the first word.
\IEEEPARstart{T}{his} demo file is intended to serve as a ``starter file''
for IEEE journal papers produced under \LaTeX\ using
IEEEtran.cls version 1.7 and later.
% You must have at least 2 lines in the paragraph with the drop letter
% (should never be an issue)
I wish you the best of success.

\appendices
\section{Particle Filter Algorithm}
Appendix one text goes here.

% you can choose not to have a title for an appendix
% if you want by leaving the argument blank
\section{}
Appendix two text goes here.


% use section* for acknowledgement
%\section*{Acknowledgment}

\bibliographystyle{IEEEtran}
\bibliography{IEEEabrv,./library}

\end{document}
