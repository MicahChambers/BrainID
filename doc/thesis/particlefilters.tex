\chapter{Particle Filters}
\label{sec:Particle Filter}
\section{Introduction}
Particle filters, a type of Sequential Monte Carlo (SMC) methods,
are a powerful way of estimating the posterior probability distribution
of a set of parameters give a timeseries of measurements. Unlike Markov 
Chain Monte Carlo estimation, Sequential Monte-Carlo methods are designed
to be used with parameters that vary with time. Unlike variations of the
Kalman filter, particle filters do not make the assumption that noise
is Gaussian. Thus particle filters are often the best solution to bayesian 
tracking for non-linear, non-gaussian systems. 

\subsection{Model}
\label{sec:Particle Filter Model}
The idea of the particle
filter is to start with a wide mixture PDF of possible parameter sets, 
and then, as measurements come in, to weight more heavily parameter sets 
that tend to give good estimations of the measurements. The reliance on
an initial mixture PDF can introduce bias; however, this effect can be
minimized by alterring the initial weights in the mixture pdf. Of course
every gradient descent must choose starting points and it is often quite
easy to establish a reasonable range of parameters, especially when the
model being used has a physical meaning. Suppose a set or stream of measurements
are given, $\{y(t), t = 1, 2, 3, ... T\}$, where $T$ is permitted to go
 to infinity. Then the goal is to find the 
parameters, $\hat{\theta}$, and underlying state time series, $\hat{x}[0:T]$
that minimize the difference between $\hat{y}[0:T]$
and $y[0:T]$. In our case, we will assume that we know the form
of the model, which is based on first principals, and that
there is some true $\theta$ and a true time-series of underlying
state variable, $x[0:T]$ that drives $y[0:T]$. Assuming a model form 
such as we do here reduces model variance, potentially at the cost of increased
bias (or systematic) error. We will assume a basic state space model:

\begin{equation}
\dot{x}(t) = f(t, x(t), u(t), \theta, \nu_x)
\end{equation}

\begin{equation}
y(t) = g(t, x(t), u(t), \theta, \nu_y)
\end{equation}

Where $x(t)$ is a vector of state variables, $\theta$ is a vector of system
constants, $u(t)$ is a stimulus, $y(t)$ an observation, and
$\nu_x$ and $\nu_y$ are random variates. Obviously any one of these could
be a vector, so for instance $u(t)$ could encode multiple types of stimuli.

Although not strictly necessary for particle filters, we will make a few
assumptions based on the particular type of systems faced in biological 
processes. First, the systems are assumed to be time invariant. This 
assumption is based on the idea that if you froze the system for $\Delta t$
seconds, when unfrozen the system would continue as if nothing happend. 
Few biological systems are predictable enough for them to be summarized
by a time varying function. Even systems that one might assume work in that
way are actually much more complicated. The heart may seem like an obvious
exception, however, the period between heartbeats vary often enough that prediction
would necessitate another state-space model. In short, we
assume no parameters are time varying, because not enough information exists to
describe any of theme in that way. Luckily particle filters are capable 
of dealing with non-white, non-Gaussian noise, so unanticipated influence
may be re-factored as noise. Secondly we assume that input cannot directly
influence the output, which in the case of the BOLD signal is a good assumption.
%Third, through some sort of preprocessing, we will assume that $\nu_d$ can be
%decreased or completely removed, since it 
Third, we will assume noise is additive, and that $\nu_x$ may be projected into
a Weiner, or other summing process that is additive with $g$ and $\nu_y$, which
will be named $\nu_d$.
Finally, $x(t)$ will encapsulate $\theta$, the unknown model constants, which
means that the vector $\dot{x}$ will always have members
that are 0. The results of these assumptions are a simplified version of the
state space equations:

\begin{equation}
\dot{x}(t) = f(x(t), u(t))
\label{eq:stateass}
\end{equation}

\begin{equation}
y(t) = g(x(t)) + \nu_y + \nu_d
\label{eq:measass}
\end{equation}

Because $\nu_d$ is something akin to and additive Weiner process $y[0:T]$, it 
will include low frequency noise. $\nu_y$ on the other hand will cause i.i.d. noise
in $y[0:T]$. For some of the tests, I will use de-trending methods to reduce the effects of 
$\nu_d$, the remainder of which will be re-factored into $\nu_y$. Both $\nu_d$ and $\nu_y$
have biological and non-biological sources. MR can lead to both types of noise, 
as demonstrated in \cite{Smith1999}. Meanwhile changes in metabolism, heart rate, or
other biochemical intervention could all lead to either $\nu_d$ or $\nu_y$.

\subsection{Prior}

The goal of the particle filter is to evolve a probably distribution 
$Pr(\hat{x}(T) | u[0:T], y[0:T])$,
that asymptotically approaches the probability distribution $Pr(x(T) | u[0:T])$.
Considering that $y$ contains measurement noise as well as noise from $x$,
 it is clear that $Pr(x(t) | u[0:T])$ is not a single true value
but a true posterior. 
To begin with, the particle filter starts with a prior distribution, and $N_p$
particles need to be drawn from that distribution, $\alpha(x)$:
\begin{equation}
\{\hat{Pr}x_i(0),w_i] : x_i(0) \sim \alpha(x), w_i = \frac{1}{N_p}, i \in \{1, 2, ... , N_p\} \}
\end{equation}

Where $N_p$ is the number of particles or points used to describe the prior 
using a Mixture PDF. 
\begin{equation}
\hat{Pr}(x(0) = \hat{x}) = \sum_{i=1}^{N_p} w_i\delta(\hat{x} - x_i(0) ) dx
\end{equation}
Where $\delta(x-x_0)$ is 1 if and only if $x = x_0$ (the Kronecker delta function).

If a true prior is preferred, then the weights
should all be $1/N_p$, and since $x_i$ was drawn from the prior, this will
be an approximation of the prior distribution. If a relatively flat prior is 
preferred, then each particle's weight could be divided by the density, $\alpha(x_i)$,
which creates a flat prior with support points in the region of $\alpha(x)$. Either
way, $\alpha(x)$ should be much broader than the true posterior, $Pr(x(0))$, since the
choice of support points is crucial to the convergence of any sampling importance
algorithm. 
For the BOLD signal all the parameters have been studied and have relatively well
known mean and variance, so a prior could be very helpful. We ran simulations for
both normalized and un-normalized priors, although we believe in cases such as this,
where a good prior exists, it should be used. 
For strictly positive parameters (members of $x$) we used a gamma distribution,
whereas for parameters that could be negative, we used a Gaussian distribution. In
both cases standard deviations twice that found in previous studies were used.

Note that all the probabilities implicitly depend on $u[0:T]$, so those terms 
will be left off for simplicity.
Once the probability, $\hat{Pr}(x(T) | x[0:T-1], y[0:T-1])$ has been found
(initially this is just Mixture approximating the prior since no measurements are 
available and no previous probabilities are available), its possible to approximate
the probability for short times between times when measurement is available, by shifting
the probability according the progression of the state equations. This is only 
an approximate, since integrating $\nu_d$ should increase uncertainty as
time without a measurement passes. 

\begin{equation}
\hat{Pr}(x(T+\Delta t)) \approx 
\sum_{i=1}^{N_p} w_i\delta\left(x - (x_i(T) + \int_T^{T+\Delta} \dot{x}_i(t) dt) \right)
\end{equation}

\subsection{Weighting}
When a measurement becomes available it is incorporated into the probability.
This process of incorporating new data is called sequential importance sampling,
and eventually causes the probability to converge. The weight is defined
as

\begin{equation}
w_i(T) \propto \frac{\hat{Pr}(x_i[0:T] | y[0:T])}{q(x_i[0:T] | y[0:T])}
\label{eq:weightfunc}
\end{equation}

where $q$ is called an \emph{importance density}, meaning it decides where
the support points for $x(T)$ are located. To remove the bias due to the
location of the support points, we divide by $q(x_i[0:T] | y[0:T])$. By dividing by 
the posterior density of the support points (particles), the effect of the particle 
distribution may be removed from the posterior density. As a result the weight
is dependent solely based
on $\hat{Pr}(x_i[0:T] | y[0:T])$, the probability of the $i^{th}$ particle's measurements
being different from $y[0:T]$ due to noise alone.
An example of an importance density would be drawing a large
number of points from the standard normal, $N(0,1)$ and then weighting each point, $l$ by 
$1/\beta(l), \beta \sim N(0,1)$. Of course if there is a far off peak in
the posterior that $q$ does not allocate support points in, there will 
be a quantization error, and that part of the density can't be modeled. This is why
it is absolutely necessary that $q$ covers $\hat{Pr}(x_i[0:T] | y[0:T])$.

$q(x_i[0:T] | y[0:T])$ may be simplified by assuming that $y(T)$ doesn't contain 
any information about $x(T-1)$, which is more practical since knowledge of future
measurements is impractical. 

\begin{eqnarray}
q(x[0:T] | y[0:T]) & = & q(x(T) | x[0:T-1], y[0:T])q(x[0:T-1] | y[0:T]) \nonumber \\
& = & q(x(T) | x[0:T-1], y[0:T])q(x[0:T-1] | y[0:T-1]) \nonumber \\
& = & q(x(T) | x(T-1), y[0:T])q(x[0:T-1] | y[0:T-1])
\end{eqnarray}

In this paper we will use 
$q(x_i(T) | x_i(T-1), y[0:T]) =  \hat{Pr}(x_i(T) | x_i(T-1))$,
based on the Markov assumption, and the belief that the state space model is 
able to approximate the true state. This means that prior to re-weighting 
particles, the particles will be distributed the same as the previous time but
moved forward according to the integration of $f(x(t), u(t))$.

In addition to $q(x_i(T) | x_i[0:T-1], y[0:T])$, the weight is also based on $Pr(x_i[0:K] | y[0:K])$,
which may be broken up as follows.
\begin{eqnarray}
\hat{Pr}(x[0:T] | y[0:T]) & = & \frac{\hat{Pr}(y[0:T], x[0:T])}{\hat{Pr}(y[0:T])} \nonumber \\
 & = & \frac{\hat{Pr}(y(T), x[0:T] | y[0:T-1]) \cancel{\hat{Pr}(y[0:T-1])}}{\hat{Pr}(y(T) | y[0:T-1]) \cancel{\hat{Pr}(y[0:T-1])}} \nonumber \\
 & = & \frac{\hat{Pr}(y(T)| x[0:T], y[0:T-1]) \hat{Pr}(x[0:T] | y[0:T-1])}{\hat{Pr}(y(T) | y[0:T-1]) } \nonumber \\
 & = & \frac{\hat{Pr}(y(T)| x[0:T], y[0:T-1]) \hat{Pr}(x(T) | x[0:T-1], y[0:T-1]) \hat{Pr}(x[0:T-1] | y[0:T-1])}{\hat{Pr}(y(T) | y[0:T-1])} \nonumber \\
\end{eqnarray}

Using the assumption that $y(t)$ is fully constrained by $x(t)$ \autoref{eq:measass},
and that $x(t)$ is fully constrained by $x(t-1)$ \autoref{eq:stateass}, we are able to
make the reasonably good assumptions that:
\begin{equation}
\hat{Pr}(y(T) | x[0:T], y[0:T-1]) = \hat{Pr}(y(T) | x(T))
\end{equation}

\begin{equation}
\hat{Pr}(x(T) | x[0:T], y[0:T-1]) = \hat{Pr}(x(T) | x(T-1))
\end{equation}

Additionally, for the particle filter $y(T)$ and $y[0:T-1]$ are 
given, and therefore constant across all particles. Thus $\hat{Pr}(x[0:T] | y[0:T])$
may be simplified to:
\begin{eqnarray}
\hat{Pr}(x[0:T] | y[0:T]) & = & \frac{\hat{Pr}(y(T)| x[0:T], y[0:T-1]) \hat{Pr}(x(T) | x[0:T-1], y[0:T-1]) 
            \hat{Pr}(x[0:T-1] | y[0:T-1])}{\hat{Pr}(y(T) | y[0:T-1])} \nonumber \\
& = & \frac{\hat{Pr}(y(T)| x(T)) \hat{Pr}(x(T) | x(T-1)) \hat{Pr}(x[0:T-1] | y[0:T-1])}{\hat{Pr}(y(T) | y[0:T-1])} \nonumber \\
& \propto & \hat{Pr}(y(T)| x(T)) \hat{Pr}(x(T) | x(T-1)) \hat{Pr}(x[0:T-1] | y[0:T-1])
\end{eqnarray}

Plugging these simplifications into \autoref{eq:weightfunc} leads to:
\begin{eqnarray}
w_i(T) & \propto & \frac{\hat{Pr}(y(T)| x(T)) \cancel{\hat{Pr}(x(T) | x(T-1))} \hat{Pr}(x[0:T-1] | y[0:T-1])}
                         {\cancel{\hat{Pr}(x_i(T) | x_i(T-1))}q(x[0:T-1] | y[0:T-1])} \nonumber \\
& \propto & w_i(T-1)\hat{Pr}(y(T)| x(T)) 
\label{eq:weightevolve}
\end{eqnarray}

Thus, by making the following relatively weak assumptions, evolving a posterior
density  is easy and requires almost no knowledge of noise distribution.
\begin{enumerate}
\item $f(t, x(t), u(t)) = f(x(t), u(t))$ and $g(t, x(t), u(t)) = g(x(t))$ provide 
a sufficiently flexible model to encapsulate the true time series.
\item $E[\nu_d] = 0$ and $E[\nu_y] = 0$, and $\nu_x = d\nu_d$, $\nu_y$ are stationary
\item The PDF $q(x_i(0))$ (the prior) fully covers $Pr(x_i(0))$
\item Markov Assumption: $Pr(x(T) | x[0:T]) = Pr(x(T) | x(T-1))$
\item $q(x[0:T-1] | y[0:T]) = q(x[0:T-1] | y[0:T-1])$
\end{enumerate}

\subsection{Basic Particle Filter Algorithm}
From the definition of $w_i$, the algorithm sequential importance sampling (SIS) 
is relatively simple. 

\begin{algorithmic}
\STATE Initialize $N_p$ Particles: 
        $\{x_i(0),w_i(0) : x_i(0) \sim \alpha(x), w_i(0) = \frac{1}{N_p}, i \in \{1, 2, ... , N_p\} \}$
\STATE $T$ = \{Set of Measurement Times\}
\FOR{$t$ in $T$}
    \FOR{$i$ in $N_p$}
        \STATE $x_i(t) = x_i(t-1) + \int_{t-1}^t f(x(\tau), u(\tau)) d\tau $
        \STATE $w_i(t) = w_i(t-1)\hat{Pr}(y(t) | x(t))$
    \ENDFOR
\ENDFOR

\STATE At $t + \Delta t$, $t \in T$, $\hat{Pr}(x(t+\Delta t)) \approx 
\sum_{i=1}^{N_p} w_i(t)\delta\left(x - (x_i(t) + \int_t^{t+\Delta t} f(x(\tau), u(\tau)) d\tau) \right)$
\end{algorithmic}
The result is then a discrete approximation of the posterior distribution. 

\subsection{Resampling}
\label{sec:Particle Filter Resampling}
As a consequence 
of the wide prior distribution (required for a proper discretization of a continuous
distribution), there will be many particles with insignificant weights. While this does help
describe the tails of the distribution very well, it means that only a small portion of the
computation will be spent describing the most probable region. Ideally every particle would 
equally decrease the entropy of the distribution, thus the lower the variance of the weights,
the more efficiently the discrete distribution is in describing the continuous distribution. 
A common measure of "Particle Degeneracy" is the effective number of particles, described
in (Bergman "Navigation and Tracking Applications", 1999, J S Liu and R Chen "Sequential 
Monte Carlo Methods for Dynamical Systems", 1998), which is based on the "true weight"
of each particle. Of course the true weight is unknown, so a heuristic approximating 
$N_{eff}$ is used:
\begin{equation}
\hat{N}_{eff} \approx \frac{N_p}{\sum_{i=1}^{N_p} w_i^2}
\label{eq:neff}
\end{equation}
Any quick run of a particle filter will reveal that unless the prior is particularly accurate,
$N_{eff}$ drops precipitously.  To alleviate this problem
a common technique known as resampling must be applied. The idea of re-sampling is to 
draw from the approximate posterior, thus generating a replica of the posterior with 
a support more suited to the distribution. Thus, if weights are all set to $1/N_p$, and 
$N_p$ points are drawn from the posterior,
\begin{equation}
\hat{\chi}_j \sim \left(\sum_{i=1}^{N_p} w_i(t)\delta(x - x_i(t))\right), j \in \{1, ..., N_p\}
\end{equation}
then $\hat{\chi} \sim \hat{x}$ should hold. Unfortunately, this isn't necessarily the truth: since the support is
still limited to the original particles, the number of unique particles can only go down.
This effect, often dubbed "particle impoverishment" can result in excessive quantization
errors in the final distribution. However, there is a solution. Instead of sampling from the
discrete distribution, a smoothing kernel is applied, and $\hat{\chi}_j$ are drawn from
that distribution. Because the distribution is continuous, there is no way for a collapse
of the particles to occur. The question then, is how to decide on the smoothing kernel. 
Often times the easiest way to sample from the continous distribution is to break the 
re-sampling down into two steps. First a member of the discrete distribution is randomly
selected based on the weights, and then based on the smoothing a nearby state variable 
is selected. The process of the selection will be defined as:
\begin{equation}
\chi_i = x_i + h\sigma \epsilon
\end{equation}
Where $h$ is the bandwidth, $\sigma$ is the standard deviation such that $\sigma \sigma^T = cov(x)$
and $\epsilon$ is drawn from the chosen kernel.
It has been proven that when all the elements of the mixture
have the same weight, as is the case after basic resampling, the kernel that minimizes the 
MSE between the estimated and true posterior is the Epanechnikov Kernel (cite Improving Regularised
Particle Filters, C Musso, N Oudjane and F LeGrand). 
\begin{equation}
K = \left\{
\begin{array}{lr}
\frac{n_x+2}{2c_{n_x}}(1-\|x\|^2) & if\ \|x\| < 1\\
0 & otherwise
\end{array}\right.
\end{equation}
%<more here>

\subsection{Weighting Function}
Because $\hat{Pr}(y(T) | x(T))$, what I will call the weighting function,
is based on an unknown distribution, it is necessary to decide on a function
that will approximate $\hat{Pr}(y(t) | x(T))$. Obviously the function, $\omega(y(t), f(x(t))$
needs to be centered at zero and have a scale comparable to the signal levels.
Obviously if the actual noise present in $y(t)$ were to be known, then that would
be the best distribution for $\Omega$. In that case, particles that fell
far out on that distribution would be statistically impossible representations
of the system, and it would be completely reasonable to throw such particles away.
While a Gaussian function is the natural choice, because this distribution
and weight are unknown we wanted to try distributions
with wider tails, so that outliers don't completely destroy particle's weights
(and thus convergence proceeds more slowly).

Another natural choice might be one of the robust estimator weight functions, for
example the Huber or bi-square. For the purpose of this work we will stick
with long tailed distributions, however it is worth noting that long tails
may not be the optimal choice in all, or even this situation. The justification
for long tailed distributions is that we believe the noise to be long tailed,
and the variance of the noise is not well known.

Therefore, we tried three weighting functions based on three distributions: Gaussian, 
Laplace and the Cauchy. The standard deviation of the distribution is extremely
important to the convergence of particle filter. A standard deviation that is 
too large will not allow the distribution to converge in any reasonable number of 
measurements. A standard deviation below the standard deviation of the noise 
will cause the algorithm to throw out perfectly acceptable particles. 
The weighting function ultimately will shape the output distribution, $P[y]$, into that
distribution, however the distribution of $x$ will still approach a reasonable
estimate of its true distribution. Even an overly wide weighting function, will
allow the Gaussian Mixture estimate of $X$ to converge to the correct location 
parameters of the "real" posterior distribution, though the scale parameters may 
be overly large.

A reasonable method of setting the standard deviation of $\Omega$ may be by 
taking a small sample from "resting" data and using the sample standard deviation.
Since this is the first attempt at using particle filters for modeling the 
BOLD model, in this work we set the standard deviation manually at <weight standard dev>,
because it gives a more consistency and control. Of course this could be taken
further, by testing the sample data against a set of stock distributions
and choosing the best fit. This of course depends on having enough samples
to make a reasonable inference, which may not always exist. 

\section{Simple, Nonlinear Example}
A typical half wave rectifier takes a AC voltage circuit and removes
one half (say the negative half) of the signal. The resulting waveform
is still not DC, however it is then possible to use a capacitor to 
smooth the signal into something similar to DC, as shown in \autoref{fig:HalfWaveIO}.
There are other, more
complex circuits that convert the negative portion into positive and
waste less energy but but here we will keep the system simple.
Thus, let us consider a simple half wave rectifier circuit, shown in 
\autoref{fig:HalfWaveRectifier}.

The half wave rectifier circuit smoothes the gaps between high voltage
with a capacitor. Thus, when $u(t)G$ is less than $v_t$, the circuit will 
discharge the capacitor and maintain a non-zero voltage,
but when $u(t)G$ is greater than $v_t$, the output voltage will be set
by $u(t)G$ and the capacitor will charge up. We will assume a very simple
model for all the components, ignoring the complex nonlinear behavior
that can occur in a diode. 

\begin{figure}
\centering
\begin{circuitikz}[scale=2, american]
\draw
 (0,0)  node[transformer core] (T) {}
 (T.A1) -- (-1,0)
 (T.A2) -- (-1,-1.05)  to[V, v=$u(t)$] (-1, 0)
 (T.B1) -- (.5, 0) to[D, l=$v_t$] (1.5,0) to[C=$C$] (1.5, -1.05)
 (1.5, 0) -- (2.5, 0) to[R=$Rm$, v=$v_y$] (2.5, -1.05) -- (T.B2) 
 (T.base) node {G}
 (T.B1) to[open, *-*, v=$V_1$] (T.B2); 
\end{circuitikz}
\caption{An Example Half Wave Rectifier Circuit, where $G$ is the transformer
gain, $v_t$ is the activation voltage of the diode, $u(t)$ is the input at time $t$, 
$C$ is the capacitance, $R$ is the load resistance and $v_y$ is the output voltage}
\label{fig:HalfWaveRectifier}
\end{figure}

\begin{figure}
\centering
\caption{Example Input/Output of the Half Wave Rectifier}
\label{fig:HalfWaveIO}
\end{figure}

Although rectifiers are typically thought
of as receiving an AC circuit 60 Hz, we will ignore such specifics and 
assume the voltage across the output of the transformer is simply a scalar
multiple of the input voltage.  As discussed in the \autoref{sec:Particle Filter Model}
any variable with uncertainty must be part of the state variable. Therefore
the state variable will be: $X(t) = \{G, v_t, C, R_m, v_y\}$. Of course,
$u(t)$ cannot be allowed to be a square wave in such a system, since that
such a signal would never get across the transformer and regardless it
would necessitate an unrealistic infinite current across the capacitor.
The state equations would then be 

\begin{equation}
v_y(t)  = f(v_y(t-1, u(t)) =  \begin{cases} 
        u(t)G & \text{ if }  u(t)G-v_y \ge v_t\\
        v_y(t-1)\left(1 - \frac{\delta t}{R_mC}\right) & \text{ if }  u(t)G-v_y < v_t
    \end{cases} 
\end{equation}

To run the particle filter is relatively easy then, since there exists 
a recursive definition of the dynamic state variable, $v_y$. To start
with, an initial distribution must be assumed and while at first a 
Gaussian seems like a good idea, all the static state variables are strictly
positive and thus not well suited to the Gaussian. In this case then,
it would be wise to start with a Gamma distribution, and just be wary of
any standard deviation that gets larger than the prior mean. We will define
the gamma distribution as follows:

\begin{equation}
X \sim Gamma(k, \theta) \rightarrow f(x) = x^{k-1}\frac{e^{-x/\theta}}{\theta^k\Gamma(k)}
\end{equation}

where $\Gamma$ is the gamma function.
The in some ways the margin for error is decided by the weighting function, which
here will define as $W(V_y, v_{yi})$, where $V_y$ is the actual measurement, $v_y$ is the 
estimate based on all the particles, and 
$v_{yi}$ is the estimate by a particular (i$^{\text{th}}$) particle. The choice of this function is difficult,
and although the Gaussian is typically used, we found the exponential helpful
in dealing with particle deprivation.  The algorithm will then look like the following,

\begin{algorithmic}
\STATE Initialize $N_p$ Particles:
\FOR{$i$ in $N_p$}
    \STATE $G \sim Gamma(\frac{\mu^2_G}{\sigma^2_G}, \frac{\sigma^2_G}{\mu_G})$
    \STATE $v_t \sim Gamma(\frac{\mu^2_{v_t}}{\sigma^2_{v_t}}, \frac{\sigma^2_{v_t}}{\mu_{v_t}})$
    \STATE $C \sim Gamma(\frac{\mu^2_C}{\sigma^2_C}, \frac{\sigma^2_C}{\mu_C})$
    \STATE $R_m \sim Gamma(\frac{\mu^2_R}{\sigma^2_R}, \frac{\sigma^2_R}{\mu_R})$
    \STATE $v_y = 0$, (Assume the system has been off for a long time)
    \STATE let $X_i(0) = \{G, v_t, C, R_m, v_y\}$
    \STATE let $w_i(0) = 1$ or to make a flat prior, $w_i(0) = \frac{1}{Pr(X_i(0))}$ 
\ENDFOR
\STATE Run the Filter:
\FOR{$t$ in Set of Measurement Times}
    \FOR{$i$ in $N_p$}
        \STATE $v_{yi}(t) = f(v_{yi}(t-1), u(t))$
        \STATE (All other members of $X_i(t)$ remain the same)
        \STATE $w_i(t) = w_i(t-1)W(V_y(t), v_y(t))$
    \ENDFOR
\ENDFOR
\end{algorithmic}

Initially the particles will have the same output, $0$, however, as $u(t)$
changes, the response of each particle to that input will result in different
outputs. Particles that have a $v_{yi}$ near $V_y$ will be weighted higher,
and others farther away will be weighted lower. As the particle filter
runs, weights will compound resulting in a distribution that asymptotically
approaches the true joint distribution of the $X(t)$.  Of course, as we
mentioned in \autoref{sec:Particle Filter Resampling}, particles weighted zero do not significantly
contribute to the empirical distribution, so re-sampling may be necessary.
If the noise is assumed to be Gaussian then it is possible to further optimize. 
Thus we let $h$ be defines as:
\begin{eqnarray}
h = [N_s8c^{-1}_{n_x}(n_x + 4)(2\sqrt{\pi})^{n_x}]^{\frac{1}{n_x +4}}
\end{eqnarray}
and although it is very possible the underlying noise is non-gaussian, the Gaussian
may work, but sub-optimally. It has been proposed that (Monte Carlo Approximations for
General State-Space Models, markus Hurzeler and Hans R. Kunsch) if the underlying 
distribution is non-Gaussian, then using this bandwidth will oversmooth. 
In reality over smoothing
should not be too great an issue because the smoothing is only being applied to find new
particles. If the distribution is over smoothed then the algorithm may not converge as rapidly;
however, because the bandwidth is still based on particle variance, which will decay as 
particles are ruled out, it is still able to converge. In fact over smoothing is preferrable
to under smoothing, since the latter would result in false negatives, but the previous only
results in a slower decay of the variance. 
At the same time, as $n_x$, the number of dimensions in
$x$, goes to infinity, the standard deviation based approximation becomes less effective
(cite a Tutorial on Particle Filters for on-line non-linear non-gaussian bayesian
tracking, sanjeev arulampalam, simon maskell, neil gordon...).  Because of the high dimensionality of our system,
and limited measurements, it is helpful to have a broader bandwidth to explore the distribution. 
Nevertheless, because 
of the potentially wide smoothing factor applied by regularized resampling, performing this
step at every measurement would allow particles a great deal of mobility. This mobility is
the enemy of convergence, which is why regularized resampling should only be done when
$\hat{N}_{eff}$ drops very low (say less than 50). Other than the periodic regularized
resampling then, the regularized particle filter is nearly identical to the basic sampling
importance sampling filter (SIS). 

 \begin{algorithmic}
\STATE Initialize $N_p$ Particles: 
        $\{x_i(0),w_i(0) : x_i(0) \sim \alpha(x), w_i(0) = \frac{1}{N_p}, i \in \{1, 2, ... , N_p\} \}$
\STATE $T$ = \{Set of Measurement Times\}
\FOR{$t$ in $T$}
    \FOR{$i$ in $N_p$}
        \STATE $x_i(t) = x_i(t-1) + \int_{t-1}^t f(x(\tau), u(\tau)) d\tau $
        \STATE $w_i(t) = w_i(t-1)\hat{Pr}(y(t) | x(t))$
    \ENDFOR

    \STATE Calculate $N_{eff}$ with \autoref{eq:neff}
    \IF{$N_{eff} < N_R$ (recommend $N_R = min(50, .1N_p)$ )}
        \STATE Calculate empirical $\sigma$ 
        \STATE $h = [N_s8c^{-1}_{n_x}(n_x + 4)(2\sqrt{\pi})^{n_x}]^{\frac{1}{n_x +4}}$
        \STATE Redraw particles using (stratified) basic resampling
        \FOR{$i$ in $N_p$}
            \STATE Draw $\epsilon \sim K$
            \STATE $x_i = x_i + h \sigma \epsilon$
        \ENDFOR
    \ENDIF
\ENDFOR

\STATE At $t + \Delta t$, $t \in T$, $\hat{Pr}(x(t+\Delta t)) \approx 
\sum_{i=1}^{N_p} w_i(t)\delta\left(x - (x_i(t) + \int_t^{t+\Delta t} f(x(\tau), u(\tau)) d\tau) \right)$

 \end{algorithmic}

The ultimate effect of this regularized resampling is a convergence similar to simulated annealing
or a genetic algorithm. Versions of $x$ that are "fit" (give good measurements) spawn more children 
nearby which allow for more accurate estimation near points of high likelihood. 
As the variance of the estimated
$x$'s decrease, the radius in which children are spawned also decreases. Eventually the radius
will approach the width of the underlying uncertainty, $\nu_x$ and $\nu_y$.

