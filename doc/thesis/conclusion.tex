\chapter{Conclusion}
\label{sec:Conclusion}
In this paper I have proposed and demonstrated the use of a particle filter
for the estimation of BOLD model parameters. In doing so, it has been possible
to estimate the BOLD time-series and judge the quality of the estimated time-
series with respect to the true BOLD signal. As result, it is possible to use
the particle filter to localize areas where a known stimuli most directly drives
neural activation. In the future it will be possible to use the estimated states
($v,q,s,f$) to drive other models and learn more about how regions of the brain
interact. 

A limitation often reached in previous works was an inconsistency of
parameter estimates, most likely because of covariance between the model parameters.
Although individual studies got consistent results, those results often differed
widely from other similar studies. The reason for this is rather clear from the
simulation results in \autoref{sec:SimLowNoise}. There is a significant amount of
trade off between parameters to the point that a signal set of parameters
is most likely not possible to derive from the BOLD response alone. It
will therefore be beneficial to combine BOLD studies with cerebral blood
flow or cerebral blood volume studies to gain multiple more measurements and
further constrain the model. That said, the benefit of the particle filter
is that it provides a full posterior distribution at the final time step.
As such the true solution should be encoded in the particle filter's final 
distribution given priors that encompass the true parameters. This is beneficial
in two ways; first, if, after the fact, some parameter becomes known 
from outside observation, it is then possible to construct a new probability
conditional on the new observation. Secondly the results from multiple runs
may be reasonably concatenated, using the final distribution from the previous
run as the prior distribution of the next run. Rather than simply providing a 
staring point for parameters to converge from, it in fact continues convergence
from the previous stopping point. 

Although many versions of the BOLD model exist, and it is tempting to use 
more detailed models; from the results found here the issue of bias error
from the BOLD model is not the biggest concern. Clarifying the distributions
of parameters for the prior should be the first concern; currently no multi-patient
full-volume studies have been done to estimate parameters.
One future study that would be beneficial in this way  would
be an extensive study of what the priors should truly be. Although \cite{Friston2000}
gives an estimate of what is thought to be reasonable values, and later studies
published their estimate of the distributions, given the interplay between
parameters it is unlikely that these priors are true to actual distribution
that occurs in vivo. As I mentioned previously, the addition of simultaneous
flow or volume measurements are another potentially powerful way to further confine the model,
and thus deal with the elasticity of parameters. As I mentioned in 
\autoref{sec:BackgroundConclusion}, a chief advantage of using physiologically
plausible models is that such data may in fact be added with relative ease.

Automatic detection of the noise level in the signal, to get a decent wieghting
function.

In conclusion, using particle filters to estimate the BOLD response are 
a powerful method of fitting to noisy data. The technique also
holds great promise as extensible platform to build more advanced models
and techniques on top of. Integrating further information is necessary to
move beyond the traditional Statistical Parametric Mapping and moving
toward biologically and medically relevant FMRI scanning techniques. 
