\documentclass[12pt]{article}
\usepackage[left=1in,top=1in,right=1in,nohead]{geometry}

\usepackage[]{times}
\usepackage{verbatim}
\usepackage{amsmath}
\usepackage{amssymb}
\usepackage{multicol}
%\usepackage{setspace}

%\doublespacing

\newcommand{\R}{\mathbb{R}}
\newcommand{\Q}{\mathbb{Q}}
\newcommand{\N}{\mathbb{N}}
\newcommand{\Z}{\mathbb{Z}}
\newcommand{\qed}{\\$\Box$}
\newcommand{\qle}{\stackrel{?}{\le}}
\newcommand{\qeq}{\stackrel{?}{=}}
\newcommand{\closure}{\overline}
\newcommand{\intersect}{\cap}
\newcommand{\union}{\cup}
\newcommand{\nullset}{\emptyset}
\newcommand{\minus}{\ \backslash\ }


\begin{comment}

:Author: Micah Chambers
:Title: Personal Statement

\end{comment}

\author{Micah Chambers}
\title{Personal Statement}

\begin{document}

%Why are you fascinated by your research area?

%What examples of leadership skills and unique
% characteristics do you bring to your chosen field?
%What personal and individual strengths do you have
% to make you a qualified applicant?

%How will receiving the fellowship contribute to your
% career goals.
\Large
\begin{multicols}{2}
\begin{flushleft}
Micah Chambers
\end{flushleft}

\begin{flushright}
Personal Statement
\end{flushright}
\end{multicols}
\normalsize

Among biomedical research areas, Neurology is probably
audacious, and in my opinion the most interesting. While many
biological systems have a thousand or more variables, the brain
has more than 100 billion. In fact, our brains are so complex, super
computers can only simulate
1 ten-millionth of the neurons. It would take another
30 years of Moore's law to scale raw computing
power to the level of a single human brain.
So the question for Neuroscientists is this: given that we cannot
simulate an entire brain, what can we do? Surprisingly, the answer is plenty.
While neuron-level simulation is a long way off,
its possible that such an analysis may not be necessary, or even
productive. By studying small \emph{regions} of the brain, 
we can still learn a great deal about how the brain
works, and how at times it doesn't work. 

One reason that neuroscience is such an
exciting field is that it is still wide open.
For instance, we still have no way to accurately
detect small activations, no way to generically
decode activation patterns, and no standard
way to determine causality. As an engineer, problem solving
is a passion; and neuroscience certainly provides plenty of problems.
Overcoming these difficulties requires both new techniques
and new ways of applying old ones.  Designing novel methods to present
data is also crucial; the traditional "activation map" is intuitive, 
but how does one show data like causality
- or switching - in an intuitive manner that doesn't 
oversimplify? All these
problems have solutions, and it is a fascinating
endeavor to search for them. 

Neuroscience is more than interesting though, it
can actually make a difference in people's lives. 
MRI, MEG and EEG can all lead to a better 
understanding of why people suffer from various conditions.
Though the benefits aren't immediate,
the ability to diagnose and understand neurological conditions
could eventually lead to methods of prevention or treatment.
Further, many conditions long thought
to be localized may in fact stem from neurological conditions;
after all, the nervous system affects every part of the body.  
Of course neuroimaging doesn't just benefit the ill, it
benefits all of humanity by leading us to a 
greater understanding of
ourselves. Why do some people have a propensity for
alcoholism? What conditions would give one person 
photographic memory and another virtually no memory? 
The answer to these questions all reside somewhere within 
the brain, waiting for use to find them. The Chinese General,
Sun Tzu, said that if you know your enemies and you
know yourself, you will not be imperiled by a hundred battles;
this is the ultimate goal of neuroscience: to know the things
that harm us, and to know ourselves.

%\section*{Why not}
\bigskip
The fact that engineering is not an end unto itself
is actually one of the reasons I find it so fascinating. 
Engineering is about designing a solution to a problem, but this
often requires learning a new field well enough
to propose solutions for it. Throughout my career, I have
attempted to broaden my horizons: as an undergraduate
I minored in
math, economics and computer science because I enjoyed considering
different approaches to problems. When I was working 
as a computer engineer in Maryland, I took a 6 hour course
in Japanese at George Mason University, something 
I hope to take up again when my schedule permits it.
In Egypt, I took an Egyptian studies course;
learning about the countries' long history, its people
and even it's politics. When working in a field as diverse as Biomedical
Engineering, it is important to be have broad experiences
because at any point these experiences may become relevant.
For instance, many of the analysis tools used for neural
connectivity and
activation originated in economics, and neural networks were
first trained not by software engineers but by sociologists.

%\section*{teamwork}
\bigskip
Last summer, I also took a robotics course in Egypt (through
Virginia Tech), and
halfway through the semester our team had still 
not been given the instructions for our semester project.
Finally, as a team, we decided on our own final project.
As the most experienced electronics engineer on the team,
I took the lead of the electronics group. I mention this
because my experiences with that team exemplify two
important leadership qualities: cooperation and management. 

There is a definite leadership quality to cooperation; 
most people spend their entire lives as 
subordinates to \emph{someone}, but this shouldn't preclude leadership.
Willingness to speak up in a group setting, and at times partake
in vigorous discussion are important parts of teamwork. Of course
willingness to make concessions and admit fault are imperative as well.
The robotics team discussed what
type of project we wanted to complete: we debated the merits of
easier goals versus ambitious ones and we did so without a specific
leader. Being able to take part in such a team is, in of itself,
a leadership skill. I had a similar experience in my Operating
Systems class: because there were only three of us on the team,
we were able to divide responsibilities without specifying a leader.
Of course, this presupposes everyone is willing to discuss
disputes so they may be resolved properly. Success of 
such groups often depends on group dynamics, so this type of
group isn't always possible.  Thus there are times when one person
ought to be in charge. When I took the lead of the electronics
team in Egypt, I did so because everyone agreed I was the best 
equipped for the task. I generally knew how the parts went together,
and I was able to delegate tasks because of that. During my
final semester as an undergraduate, I took a Bioinformatics course,
in which we had a semester long project.
I started working on the project early 
and although I wasn't the assigned leader, I 
eventually began delegating tasks. As the only team member
with a working knowledge of the project, it was my duty to
step up.  As a TA for embedded systems last spring, I acted
as the project manager for five teams of four. In that case, I
delegated very little, but I provided the team with direction
and incentives to keep on top of their work.
In the end, the type of leadership that is necessary depends on the 
circumstances, and my diverse experience has definitely
helped me understand leadership better.

Concluding, I believe that I am uniquely qualified to 
research Biomedical Imaging. Besides taking courses in
physiology and medical imaging, I have leadership
experience which will help me work in a team,
and diverse interests which which help me learn new fields.
%There are two primary ways that this fellowship will aid me.
%First it will be 
%helpful to have access to the computing resources. Every
%type of neural analysis requires computers.
%Testing new algorithms is imperative, but it is time consuming
%even on fast machines; thus is would be useful to have
%access to the TeraGrid. 
%Secondly this fellowship would allow me to spend more
%time developing a broad knowledge base, rather than having to work to pay
%for school. As I mentioned, advances in neuroscience often come from
%diverse fields, and thus requires a diverse knowledge set.
%Studying how other fields solve similar problems is crucial
%in Bioimaging, so I prefer to read 
%not just Engineering or Neuroscience journals, but 
%economics and biology journals as well. Having this fellowship
%would certainly give me more time to do that.

\end{document}
