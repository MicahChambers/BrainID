\documentclass{article}
\usepackage[left=1cm,top=2cm,right=1cm,nohead,nofoot]{geometry}

%\usepackage{times}
%\usepackage{tikz}
%\usetikzlibrary{trees}
%\usetikzlibrary{shapes,snakes}
\usepackage{verbatim}
\usepackage{amsmath}
\usepackage{amssymb}
\usepackage{algorithmic}
\usepackage{setspace}
\usepackage{graphicx}
\usepackage{cancel}

%\doublespacing

\newcommand{\R}{\mathbb{R}}
\newcommand{\Q}{\mathbb{Q}}
\newcommand{\N}{\mathbb{N}}
\newcommand{\Z}{\mathbb{Z}}
\newcommand{\qed}{\\$\Box$}
\newcommand{\qle}{\stackrel{?}{\le}}
\newcommand{\qeq}{\stackrel{?}{=}}
\newcommand{\closure}{\overline}
\newcommand{\intersect}{\cap}
\newcommand{\union}{\cup}
\newcommand{\nullset}{\emptyset}
\newcommand{\minus}{\ \backslash\ }


\begin{comment}

:Author: Micah Chambers
\end{comment}

\begin{document}
\section{Introduction}
Functional Magnetic Resonance Imaging (FMRI) is a powerful tool in the analysis
of neural activity. Despite its rather limited temporal resolution, FMRI is still
the best way of measuring neural activity for the majority of the brain.
Whereas other methods
of analyzing neural signals can be invasive or difficult to acquire, 
FMRI is relatively quick and cheap, and its analysis straight forward.
Because of these benefits, FMRI continues to be crucial to the study of human 
cognition. Despite its prevalence, there have been relatively few developments
in the actual analysis of FMRI images. A steady stream of studies have adapted
the BOLD model from the initial Baloon model first proposed by \cite{Buxton1998}
all the way to full fully autonomous system of equations \cite{Riera2004}. And while
there have been numerous forks in the model, enough in fact to make an entire paper
studying the differences,\cite{Deneux2006}, it is widely known that all these
models have quantitatively less bias error than General Linear Model which is
typically employed today. Then again, depending on the model there may between
seven \cite{Riera2004} and fifteen \cite{Zheng2002} parameters per voxel to
be optimized. Clearly there is a significant risk of error due to variance
with so many degrees of freedom, not to mention a significantly increased
computation cost. In this thesis I demonstrate the use of a 
particle filter as a means of addressing these problems.

\subsection{FMRI}
Magnetic Resonance Imaging, MRI is a method of building 3D images
non-invasively, based on the difference between nuclear spin
relaxation times in various molecules. Initially the entity being
imaged is brought into a large magnetic field which aligns the spins
of molecules in the same direction; radio frequency (RF) signals may
then be used to excite nuclear spin away from the steady alignment. 
As the nuclei precess back to their original orientation, they resonate
at the same RF frequency of their original excitation. Conveniently, the
excitation of nuclear spins return their original state at different
rates, called the T1 relaxation time, depending on the properties of 
the material excited. Additionally, the
coherence of the spins also decay differently (and quite a bit faster
than T1) based on the properties of the region that has been excited.
This gives two primary methods of contrasting substances,
which is the basis of T1 and T2 weighted images. Additionally, there
dephasing occurs at two different rates, the T2 relaxation time,
which is impossible to recover from, and T2$^*$ relaxation, which is
much faster, but possible to recover from with an inversion pulse.
Oftentimes T1 relaxation times can be on the order of seconds if 
a significant excitation pulse is applied. 
In order to rapidly acquire entire brain images, as is done in Functional 
MRI, a single large excitation pulse is applied to the entire brain,
and the entire volume is acquired in a single T1 relaxation period. 
Because the entire k-space (spatial-frequency) volume is acquired 
from a single excitation, the signal to noise ration is very low
in this type  of imaging (Echo Planar Imaging). 

Increasing the spatial resolution of EPI imaging necessarily 
requires more time or faster magnetic field switching. Increasing
magnet switching rates though is difficult, because it can result in
more artifacts, or even lower signal to noise ratios. The result is
that at best FMRI is capable of 1.2 second temporal resolution. 
Additionally, the means that each voxel of the image will contain 
the sum of a large amount neurons, capillaries and veins. Thus, the
FMRI signal, which is sensitive to the chemical composition of 
materials, is summing up the composition of various types of tissue
in addition to the blood, whose composition is what we actually care about.
In particular, the presence of Deoxyhemoglobin, Hemoglobin whose
oxygen has been used by a metabolic process, has a larger magnetic
response compared to Oxygenated Hemoglobin. Thus, capillaries near
very active cells will typically have a higher Deoxyhemoglobin content,
and regions with lower activity will have a lower Deoxyhamoglobin content.
Unfortunately, as mentioned previously, blood is only a small part of
each voxel, which means that a single EPI image doesn't tell much about
Deoxyhemoglobin content. However assuming blood is the only thing changing
in the short term, percent difference from a baseline signal \emph{will}
tell us something. FMRI analysis is thus necessarily performed on the percent change
from the baseline. Luckily the assumption that tissue content does not change in
the short run is actually pretty good, although other facts can pollute
the baseline signal, as we will discuss later. 

\subsection{BOLD Physiology}
The physiology that leads to the BOLD model has been well studied over the 
past decade and a half. While some uncertainty remains (for instance in
relation to the BOLD post-stimulus undershoot <papers relating to back and
forth on post-stimulus undershoot>) generally the model fits reality very 
well. As with every piece of tissue in the body, the brain requires oxygen
to extract energy from glucose in the blood. This process of removing oxygen
from red blood cells sets into motion a chain of events that locally 
alters the composition of the blood. Inactive neurons can be thought of
as a slingshot cocked and ready to fire. As soon as a signal moves up
the axon and causes the neuron to fire (and thus changes the state
of a membrane at that location), ions quickly move across the altered
membrane to compensate for a high charge imbalance between the interior
and exterior of the cell. This process, similar to allowing a stretched
rubber band to contract leaves the system at a lower energy state. This cannot
last though, because the neuron needs to be ready to fire again rather quickly.
To return the cell to a ready position the membrane becomes impermeable
to the ions again, and then begins pumping ions back into the cell. This
process takes a large amount of energy, whereas the actual firing takes
very little energy. Thus, after firing, glucose is burned, removing oxygen
from the blood and thus causing a dip in the ratio of oxygenated hemoglobin to 
de-oxygenated hemoglobin. This is the first potentially measurable effect
that MRI can see, although it usually lasts fewer than 2 seconds making it
difficult for FMRI to catch. As a result of the decreased amount of local oxygen,
the capillaries compensate by increasing blood flow to that region. Because,
of the quickly increased blood flow into local capillaries, the blood
volume increases in addition to the local oxygen content. This leads to a Windkessel
effect which further lags the normalization of oxygen content. The effect of
all this is an overshoot in the oxygen content above the initial level. After
the work of recharging is done, there may be a prolonged undershoot lasting
as much as 90 seconds \cite{spatial_undershoot}, though the reason for this
is debated \cite{origin_poststim}. 

A large number of models have been proposed for the BOLD signal with varying
amounts of complexity. The simplest model is the so called "Balloon Model"
which first proposed the windkessel effect as a factor the BOLD response.
The model we will use is the model proposed by Buxton Et. Al. and later used
in Riera et. al. \cite{Riera}. The model has four state variables, $s$, $f$,
$v$, $q$, representing flow inducing signal, cerebral blood flow, cerebral
blood volume and deoxyhemaglobin to hemaglobin ratio, respectively. The 
state variables change over time, given by the state evolution equations:
\begin{equation}
\dot{s} = \epsilon u -  
\end{equation}

\begin{equation}
\dot{f} = s
\end{equation}

\begin{equation}
\dot{v} = 
\end{equation}

\begin{equation}
\dot{q} = 
\end{equation}

Additional papers have added such things as
....

All these effects are generally accepted as the cause of the BOLD signal, 
but FMRI doesn't detect this happening in one neuron, but rather as the 
aggregate over millions of cells. Though local neurons act
"together" (i.e. around the same time), the density of neurons, the
density of capillaries, and slight differences in activation across 
a particular voxel can all lead to signal attenuation or noise. 
A particularly insidious type of noise present in FMRI is a low frequncey
drift, characterized by a Weiner process. Though not present in all
regions, it is prevalent enough to cause problems \cite{detrend}. It is still not
clear where exactly this noise comes although it is possible it is 
the result of magnets heating up, or some distortion in magnetic
fields. It is clear that this drift signal is not due to a true physiological
effects however, given its presence in cadavers and phantoms\cite{drift}.

\section{Current Techniques}
\subsection{Basic Statistical Parametric Mapping}
The most basic method of analyzing FMRI data is through a basic T-test
between "resting state" and "active state" samples. This is done by 
taking the average and variance of the inactive period, and the 
active period separately then treating them both as gaussian distributions.
If they are in fact Gaussian distributions, then a basic t-test will
give the likelihood that the samples came from the same distribution
(the p-value). Of course, this test is fraught with problems; even if
the drift mentioned earlier has been removed, there is little reason
to believe that the noise is Gaussian, or even stable. Additionally, 
even if the noise were Gaussian, a t-test with a p-value of .05 over
5000 or more samples is on average going to generate $.05*5000$ false
positives. To compensate for this, bonferoni correction, also known as
multiple comparison tests are performed; essentially p-values are 
divided by the number of independent
tests being run. This, however, leads to extremely low p-values, so
low that it would be impossible for any biological system to satisfy. To
compensate, a Gaussian kernel is applied to the image, thus reducing
variance (and thus separating the active and inactive distributions)
as well as decreasing the effective number of voxels. Since t-tests are
now no longer being applied to n <I need to define n> independent voxels,
the factor by which the p-value must be divided by can be decreased.
<Do I need to mathematically define all this?> The derivation and application
of random field theory, and its use can be found in various papers \cite{univ_mult_rft}.

\subsection{General Linear Model}
The most used form of FMRI analysis is still based on Statistical Parametric
Mapping, but is able to account for several different levels or types
of stimulus. By adding a General Linear Model to the analysis, the
output signal timeseries (what FMRI detects) is regressed over the weighted
sum of the various confound's timeseries. The equation for a general linear
model is then
\begin{equation}
Y(t) = X(t)\beta + \epsilon(t)
\end{equation}
where $Y(t)$ is the smoothed or detrended timeseries of measurements,
$X(t)$ is a row vector of stimuli, $\beta$ is a column vector of weights,
and $\epsilon$ is the error. Thus for every time, the measurement is
assumed to be a weighted sum of the inputs plus some error. The calculation
of $\beta$ then is merely a gradient descent search to minimize the
mean squared error. 

<Image of GLM>

Of course, a square wave input is not going to result in a square wave
in the activation of brain regions. Thus, various methods are used to 
smooth $X(t)$ through time, and bandlimit the input. The best technique
is convolving the stimulus input with a hemodynamic response function,
which mimicks the basic shape of BOLD activation, including a delay
due to rise time and fall time. This hemodynamic signal is static however,
so every region of the brain gets the same design matrices (X(t)), 
although the weights of various stimulus or confounds are allowed to vary. 

<Image of Hemodynamic Response Function>

Ultimately, activation due to a particular stimuli is decided by the 
$\beta$ value corresponding to that stimuli's column of $X(t)$.
<Need to check this>
The null hypothesis as to whether the outcome was random is then
based on a t-test of the $\epsilon(t)$ timeseries. 

\subsection{Whats wrong with these techniques}
There are a few problems with the techniques mentioned in the 
previous sections. First, they essentially ignore prior knowledge about
the system. Although the most advanced form of the general linear model includes
a "Hemodynamic Response Function," that hemodynamic response function is
static across every region of the brain. It is well known that capillary beds
are not uniform and so blood perfusion cannot possible be static across the
brain. Thus, if extra information were available a-priori, that information
could not be incorporated without modifications to the General Linear Model.
Similarly heart rate could not be added either. It would obviously be advantageous
to have true physiological parameters as entry points for these various other
model parameters. The physiological models for the BOLD signal are quite good
and based on realistic physics. While the exact connection between a stimulus
and the flow inducing signal is not precisely known, model fits are actually
quite good \cite{nonlinearmodels}. Regardless, being based on some real 
physiological 
parameter would allow for the establishment of reasonable priors and decrease
model variance without breaking a sweat. Of course, using real parameters 
has the additional
bonus of potentially providing information about physical pathologies. It
is quite possible that physical properties such as decreased compliance of
blood vessels could indicate a neurological condition that is not easily
seen in a T1 or T2 map. In essence, this could make FMRI a much more 
useful clinical tool than it is now. The other problem with linear models
is that they are a linear fit to a nonlinear signal. It is not uncommon
for data to be thrown out in FMRI studies because no significant activation
has been seen. However, if, for whatever reason, the BOLD response
was acting more nonlinear than in other patients it would be completely
possible for SPM to miss that activation. 

<Image with two different $\alpha$s>

<image comparing the results of 10\% changes in various signals>

Secondly, these methods are still based on t-tests, which notoriously
lack robustness to non-Gaussian noise. While different techniques 
exist for imposing Gaussianity, those techniques are incapable of
discriminating noise from signal. There is no way to know how much 
signal is removed by various smoothing techniques, or even if entire 
regions have been smoothed into oblivion. Instead of extensively filtering data to 
remove noise, the analysis method itself must be robust a wide range of noise,
which is why we propose here the use of particle filters.

\section{Proposed Approach}
\subsection{Goal}
The ultimate goal of this project is to provide a new set of tools
for analyzing FMRI data. Whereas SPM techniques have been highly 
successful at finding macroscopic regions of activation, linear 
modeling can carry significant bias error due to lack of model
flexibility. While adding parameters can significantly increase
error due to model variance, this effect is mitigated by the fact
that we plan to use a model that is based on first principals. The
purpose of this paper is thus to evaluate the potential of using
a particle filter along with the BOLD model to derive physical 
parameters. In so doing, we hope to be able to show that neuronal
efficacy, $\epsilon$ is a suitable variable for estimating voxel 
activation from a standard FMRI image. We also hope to show that 
estimated posterior distribution of the parameters, derived from
the particle filter, is able to provide an accurate measure of the
confidence interval.

\subsection{Introduction to Particle Filters}
Particle filters, a type of Sequential Monte Carlo (SMC) methods
are a powerful way of estimating the posterior probability distribution
of a set of parameters give a timeseries of measurements. Unlike Markov 
Chain Monte Carlo estimation, Sequential Monte-Carlo methods are designed
to be used with parameters that vary with time. Unlike variations of the
Kalman filter, particle filters do not make the assumption that noise
is Gaussian. Thus particle filters are often the best solution to bayesian 
tracking for non-linear, non-gaussian systems. 
%todo: might need to reference attempts to model a system of our type with
%unscented kalman filter
\subsubsection{Model}
The idea of the particle
filter is to start with a wide mixture PDF of possible parameter sets, 
and then, as measurements come in, to weight more heavily parameter sets 
that tend to give good estimations of the measurements. The reliance on
an initial mixture PDF can introduce bias; however, this effect can be
minimized by alterring the initial weights in the mixture pdf. Of course
every gradient descent must choose starting points and it is often quite
easy to establish a reasonable range of parameters, especially when the
model being used has a physical meaning. Suppose a set or stream of measurements
are given, $\{y(t), t = 1, 2, 3, ... T\}$, where $T$ is permitted to go
 to infinity. Then the goal is to find the 
parameters, $\hat{\theta}$, and underlying state time series, $\hat{x}[0:T]$
that minimize the difference between $\hat{y}[0:T]$
and $y[0:T]$. In our case, we will assume that we know the form
of the model, which is based on first principals, and that
there is some true $\theta$ and a true time-series of underlying
state variable, $x[0:T]$ that drives $y[0:T]$. Assuming a model form 
such as we do here reduces model variance, potentially at the cost of increased
bias (or systematic) error. We will assume a basic state space model:

\begin{equation}
\dot{x}(t) = f(t, x(t), u(t), \theta, \nu_x)
\end{equation}

\begin{equation}
y(t) = g(t, x(t), u(t), \theta, \nu_y)
\end{equation}

Where $x(t)$ is a vector of state variables, $\theta$ is a vector of system
constants, $u(t)$ is a stimulus, $y(t)$ an observation, and
$\nu_x$ and $\nu_y$ are random variates. Obviously any one of these could
be a vector, so for instance $u(t)$ could encode multiple types of stimuli.

Although not generally necessary for particle filters, we will make a few
assumptions based on the particular type of systems faced in biological 
processes. First, the systems are assumed to be time invariant. This 
assumption is based on the idea that if you froze the system for $\Delta t$
seconds, when unfrozen the system would continue as if nothing happend. 
Few biological systems are predictible enough for them to be summarized
by a time varying function. Although the heart may seem like an obvious
exception, period between heartbeats vary often enough that prediction
would necessate another state-space model. In short, we
assume no parameters are time varying, because not enough information exists to
describe any of theme in that way. Luckily particle filters are capable 
of dealing with non-white, non-Gaussian noise, so unanticipated influence
may be re-factored as noise. Secondly we assume that input cannot directly
influence the output, which in the case of the BOLD signal is a good assumption.
%Third, through some sort of preprocessing, we will assume that $\nu_d$ can be
%decreased or completely removed, since it 
Third, we will assume noise is additive, and that $\nu_x$ may be projected into
a weiner, or other summing process that is additive with $g$ and $\nu_y$, which
will be named $\nu_d$.
Finally, $x(t)$ will encapsulate $\theta$, the unknown model constants, which
means that the vector $\dot{x}$ will always have members
that are 0. The results of these assumptions are a simplified version of the
state space equations:

\begin{equation}
\label{stateass}
\dot{x}(t) = f(x(t), u(t))
\end{equation}

\begin{equation}
\label{measass}
y(t) = g(x(t)) + \nu_y + \nu_d
\end{equation}

Because $\nu_d$ is something akin to and additive Weiner process $y[0:T]$, it 
will include low frequency noise. $\nu_y$ on the other hand will cause i.i.d. noise
in $y[0:T]$. For some of the tests, I will use de-trending methods to reduce the effects of 
$\nu_d$, the remainder of which will be re-factored into $\nu_y$. Both $\nu_d$ and $\nu_y$
have biological and non-biological sources. MR can lead to both types of noise, 
as demonstrated in \cite{drift}. Meanwhile changes in metabolism, heart rate, or
other biochemical intervention could all lead to either $\nu_d$ or $\nu_y$.

\subsubsection{Prior}

The goal of the particle filter is to evolve a probably distribution 
$Pr(\hat{x}(T) | u[0:T], y[0:T])$,
that asymptotically approaches the probability distribution $Pr(x(T) | u[0:T])$.
Considering that $y$ contains measurement noise and noise in $x$ can drive
changes in $y$, it is clear that $Pr(x(t) | u[0:T])$ is not a single true value
but a true posterior. 
To begin with, the particle filter starts with a proposal distribution, and $N_p$
particles need to be drawn from that distribution, $\alpha(x)$:
\begin{equation}
\{\hat{Pr}x_i(0),w_i] : x_i(0) \sim \alpha(x), w_i = \frac{1}{N_p}, i \in \{1, 2, ... , N_p\} \}
\end{equation}

Where $N_p$ is the number of particles or points used to describe the prior 
using a Mixture PDF. 
\begin{equation}
\hat{Pr}(x(0) = \hat{x}) = \sum_{i=1}^{N_p} w_i\delta(\hat{x} - x_i(0) ) dx
\end{equation}
Where $\delta(x-x_0)$ is 1 if and only if $x = x_0$ (the Kronecker delta function).

If a true prior is preferred, then the weights
should all be $1/N_p$, and since $x_i$ was drawn from the prior, this will
be an approximation of the prior distribution. If a relatively flat prior is 
preferred, then each particle's weight could be divided by the density, $\alpha(x_i)$,
which creates a flat prior with support points in the region of $\alpha(x)$. Either
way, $\alpha(x)$ should be much broader than the true posterior, $Pr(x(0))$, since the
choice of support points is crucial to the convergence of any sampling importance
algorithm. 
For the BOLD signal all the parameters have been studied and have relatively well
known mean and variance, so a prior could be very helpful. We ran simulations for
both normalized and un-normalized priors, although we believe in cases such as this,
where a good prior exists, it should be used. 
For strictly positive parameters (members of $x$) we used a gamma distribution,
whereas for parameters that could be negative, we used a Gaussian distribution. In
both cases standard deviations twice that found in previous studies were used.

Note that all the probabilities implicitly depend on $u[0:T]$, so those terms 
will be left off for simplicity.
Once the probability, $\hat{Pr}(x(T) | x[0:T-1], y[0:T-1])$ has been found
(initially this is just Mixture approximating the prior since no measurements are 
available and no previous probabilities are available), its possible to approximate
the probability for short times between times when measurement is available, by shifting
the probability according the progression of the state equations. This is only 
an approximate, since integrating $\nu_d$ should increase uncertainty as
time without a measurement passes. 

\begin{equation}
\hat{Pr}(x(T+\Delta t)) \approx 
\sum_{i=1}^{N_p} w_i\delta\left(x - (x_i(T) + \int_T^{T+\Delta} \dot{x}_i(t) dt) \right)
\end{equation}

\subsubsection{Weighting}
When a measurement becomes available it is incorporated into the probability.
This process of incorporating new data is called sequential importance sampling,
and eventually causes the probability to converge. The weight is defined
as

\begin{equation}
\label{weightfunc}
w_i(T) \propto \frac{\hat{Pr}(x_i[0:T] | y[0:T])}{q(x_i[0:T] | y[0:T])}
\end{equation}

where $q$ is called an \emph{importance density}, meaning it decides where
the support points for $x(T)$ are located. To remove the bias due to the
location of the support points, we divide by $q(x_i[0:T] | y[0:T])$. By dividing by 
the posterior density of the support points (particles), the effect of the particle 
distribution may be removed from the posterior density. As a result the weight
is dependent solely based
on $\hat{Pr}(x_i[0:T] | y[0:T])$, the probability of the $i^{th}$ particle's measurements
being different from $y[0:T]$ due to noise alone.
An example of an importance density would be drawing a large
number of points from the standard normal, $N(0,1)$ and then weighting each point, $l$ by 
$1/\beta(l), \beta \sim N(0,1)$. Of course if there is a far off peak in
the posterior that $q$ does not allocate support points in, there will 
be a quantization error, and that part of the density can't be modeled. This is why
it is absolutely necessary that $q$ covers $\hat{Pr}(x_i[0:T] | y[0:T])$.

$q(x_i[0:T] | y[0:T])$ may be simplified by assuming that $y(T)$ doesn't contain 
any information about $x(T-1)$, which is more practical since knowledge of future
measurements is impractical. 

\begin{eqnarray}
q(x[0:T] | y[0:T]) & = & q(x(T) | x[0:T-1], y[0:T])q(x[0:T-1] | y[0:T]) \nonumber \\
& = & q(x(T) | x[0:T-1], y[0:T])q(x[0:T-1] | y[0:T-1]) \nonumber \\
& = & q(x(T) | x(T-1), y[0:T])q(x[0:T-1] | y[0:T-1])
\end{eqnarray}

In this paper we will use 
$q(x_i(T) | x_i(T-1), y[0:T]) =  \hat{Pr}(x_i(T) | x_i(T-1))$,
based on the Markov assumption, and the belief that the state space model is 
able to approximate the true state. This means that prior to re-weighting 
particles, the particles will be distributed the same as the previous time but
moved forward according to the integration of $f(x(t), u(t))$.

In addition to $q(x_i(T) | x_i[0:T-1], y[0:T])$, the weight is also based on $Pr(x_i[0:K] | y[0:K])$,
which may be broken up as follows.
\begin{eqnarray}
\hat{Pr}(x[0:T] | y[0:T]) & = & \frac{\hat{Pr}(y[0:T], x[0:T])}{\hat{Pr}(y[0:T])} \nonumber \\
 & = & \frac{\hat{Pr}(y(T), x[0:T] | y[0:T-1]) \cancel{\hat{Pr}(y[0:T-1])}}{\hat{Pr}(y(T) | y[0:T-1]) \cancel{\hat{Pr}(y[0:T-1])}} \nonumber \\
 & = & \frac{\hat{Pr}(y(T)| x[0:T], y[0:T-1]) \hat{Pr}(x[0:T] | y[0:T-1])}{\hat{Pr}(y(T) | y[0:T-1]) } \nonumber \\
 & = & \frac{\hat{Pr}(y(T)| x[0:T], y[0:T-1]) \hat{Pr}(x(T) | x[0:T-1], y[0:T-1]) \hat{Pr}(x[0:T-1] | y[0:T-1])}{\hat{Pr}(y(T) | y[0:T-1])} \nonumber \\
\end{eqnarray}

Using the assumption that $y(t)$ is fully constrained by $x(t)$ \eqref{measass},
and that $x(t)$ is fully constrained by $x(t-1)$ \eqref{stateass}, we are able to
make the reasonably good assumptions that:
\begin{equation}
\hat{Pr}(y(T) | x[0:T], y[0:T-1]) = \hat{Pr}(y(T) | x(T))
\end{equation}

\begin{equation}
\hat{Pr}(x(T) | x[0:T], y[0:T-1]) = \hat{Pr}(x(T) | x(T-1))
\end{equation}

Additionally, for the particle filter $y(T)$ and $y[0:T-1]$ are 
given, and therefore constant across all particles. Thus $\hat{Pr}(x[0:T] | y[0:T])$
may be simplified to:
\begin{eqnarray}
\hat{Pr}(x[0:T] | y[0:T]) & = & \frac{\hat{Pr}(y(T)| x[0:T], y[0:T-1]) \hat{Pr}(x(T) | x[0:T-1], y[0:T-1]) 
            \hat{Pr}(x[0:T-1] | y[0:T-1])}{\hat{Pr}(y(T) | y[0:T-1])} \nonumber \\
& = & \frac{\hat{Pr}(y(T)| x(T)) \hat{Pr}(x(T) | x(T-1)) \hat{Pr}(x[0:T-1] | y[0:T-1])}{\hat{Pr}(y(T) | y[0:T-1])} \nonumber \\
& \propto & \hat{Pr}(y(T)| x(T)) \hat{Pr}(x(T) | x(T-1)) \hat{Pr}(x[0:T-1] | y[0:T-1])
\end{eqnarray}

Plugging these simplifications into \eqref{weightfunc} leads to:
\begin{eqnarray}
\label{weightevolve}
w_i(T) & \propto & \frac{\hat{Pr}(y(T)| x(T)) \cancel{\hat{Pr}(x(T) | x(T-1))} \hat{Pr}(x[0:T-1] | y[0:T-1])}
                         {\cancel{\hat{Pr}(x_i(T) | x_i(T-1))}q(x[0:T-1] | y[0:T-1])} \nonumber \\
& \propto & w_i(T-1)\hat{Pr}(y(T)| x(T)) 
\end{eqnarray}

Thus, by making the following relatively weak assumptions, evolving a posterior
density  is easy and requires almost no knowledge of noise distribution.
\begin{enumerate}
\item $f(t, x(t), u(t)) = f(x(t), u(t))$ and $g(t, x(t), u(t)) = g(x(t))$ provide 
a sufficiently flexible model to encapsulate the true time series.
\item $E[\nu_d] = 0$ and $E[\nu_y] = 0$, and $\nu_x = d\nu_d$, $\nu_y$ are stationary
\item The PDF $q(x_i(0))$ (the prior) fully covers $Pr(x_i(0))$
\item Markov Assumption: $Pr(x(T) | x[0:T]) = Pr(x(T) | x(T-1))$
\item $q(x[0:T-1] | y[0:T]) = q(x[0:T-1] | y[0:T-1])$
\end{enumerate}

\subsubsection{Basic Particle Filter Algorithm}
From the definition of $w_i$, the algorithm sequential importance sampling (SIS) 
is relatively simple. 

\begin{algorithmic}
\STATE Initialize $N_p$ Particles: 
        $\{x_i(0),w_i(0) : x_i(0) \sim \alpha(x), w_i(0) = \frac{1}{N_p}, i \in \{1, 2, ... , N_p\} \}$
\STATE $T$ = \{Set of Measurement Times\}
\FOR{$t$ in $T$}
    \FOR{$i$ in $N_p$}
        \STATE $x_i(t) = x_i(t-1) + \int_{t-1}^t f(x(\tau), u(\tau)) d\tau $
        \STATE $w_i(t) = w_i(t-1)\hat{Pr}(y(t) | x(t))$
    \ENDFOR
\ENDFOR

\STATE At $t + \Delta t$, $t \in T$, $\hat{Pr}(x(t+\Delta t)) \approx 
\sum_{i=1}^{N_p} w_i(t)\delta\left(x - (x_i(t) + \int_t^{t+\Delta t} f(x(\tau), u(\tau)) d\tau) \right)$
\end{algorithmic}
The result is then a discrete approximation of the posterior distribution. 

\subsubsection{Resampling}
As a consequence 
of the wide prior distribution (required for a proper discretization of a continuous
distribution), there will be many particles with insignificant weights. While this does help
describe the tails of the distribution very well, it means that only a small portion of the
computation will be spent describing the most probable region. Ideally every particle would 
equally decrease the entropy of the distribution, thus the lower the variance of the weights,
the more efficiently the discrete distribution is in describing the continuous distribution. 
A common measure of "Particle Degeneracy" is the effective number of particles, described
in (Bergman "Navigation and Tracking Applications", 1999, J S Liu and R Chen "Sequential 
Monte Carlo Methods for Dynamical Systems", 1998), which is based on the "true weight"
of each particle. Of course the true weight is unknown, so a heuristic approximating 
$N_{eff}$ is used:
\begin{equation}
\label{neff}
\hat{N}_{eff} \approx \frac{N_p}{\sum_{i=1}^{N_p} w_i^2}
\end{equation}
Any quick run of a particle filter will reveal that unless the prior is particularly accurate,
$N_{eff}$ drops precipitously.  To alleviate this problem
a common technique known as resampling must be applied. The idea of re-sampling is to 
draw from the approximate posterior, thus generating a replica of the posterior with 
a support more suited to the distribution. Thus, if weights are all set to $1/N_p$, and 
$N_p$ points are drawn from the posterior,
\begin{equation}
\hat{\chi}_j \sim \left(\sum_{i=1}^{N_p} w_i(t)\delta(x - x_i(t))\right), j \in \{1, ..., N_p\}
\end{equation}
then $\hat{\chi} \sim \hat{x}$ should hold. Unfortunately, this isn't necessarily the truth: since the support is
still limited to the original particles, the number of unique particles can only go down.
This effect, often dubbed "particle impoverishment" can result in excessive quantization
errors in the final distribution. However, there is a solution. Instead of sampling from the
discrete distribution, a smoothing kernel is applied, and $\hat{\chi}_j$ are drawn from
that distribution. Because the distribution is continuous, there is no way for a collapse
of the particles to occur. The question then, is how to decide on the smoothing kernel. 
Often times the easiest way to sample from the continous distribution is to break the 
re-sampling down into two steps. First a member of the discrete distribution is randomly
selected based on the weights, and then based on the smoothing a nearby state variable 
is selected. The process of the selection will be defined as:
\begin{equation}
\chi_i = x_i + h\sigma \epsilon
\end{equation}
Where $h$ is the bandwidth, $\sigma$ is the standard deviation such that $\sigma \sigma^T = cov(x)$
and $\epsilon$ is drawn from the chosen kernel.
It has been proven that when all the elements of the mixture
have the same weight, as is the case after basic resampling, the kernel that minimizes the 
MSE between the estimated and true posterior is the Epanechnikov Kernel (cite Improving Regularised
Particle Filters, C Musso, N Oudjane and F LeGrand). 
\begin{equation}
K = \left\{
\begin{array}{lr}
\frac{n_x+2}{2c_{n_x}}(1-\|x\|^2) & if\ \|x\| < 1\\
0 & otherwise
\end{array}\right.
\end{equation}
%<more here>

If the noise is assumed to be Gaussian then it is possible to further optimize. 
Thus we let $h$ be defines as:
\begin{eqnarray}
h = [N_s8c^{-1}_{n_x}(n_x + 4)(2\sqrt{\pi})^{n_x}]^{\frac{1}{n_x +4}}
\end{eqnarray}
and although it is very possible the underlying noise is non-gaussian, the Gaussian
may work, but sub-optimally. It has been proposed that (Monte Carlo Approximations for
General State-Space Models, markus Hurzeler and Hans R. Kunsch) if the underlying 
distribution is non-Gaussian, then using this bandwidth will oversmooth. 
In reality over smoothing
should not be too great an issue because the smoothing is only being applied to find new
particles. If the distribution is over smoothed then the algorithm may not converge as rapidly;
however, because the bandwidth is still based on particle variance, which will decay as 
particles are ruled out, it is still able to converge. In fact over smoothing is preferrable
to under smoothing, since the latter would result in false negatives, but the previous only
results in a slower decay of the variance. 
At the same time, as $n_x$, the number of dimensions in
$x$, goes to infinity, the standard deviation based approximation becomes less effective
(cite a Tutorial on Particle Filters for on-line non-linear non-gaussian bayesian
tracking, sanjeev arulampalam, simon maskell, neil gordon...).  Because of the high dimensionality of our system,
and limited measurements, it is helpful to have a broader bandwidth to explore the distribution. 
Nevertheless, because 
of the potentially wide smoothing factor applied by regularized resampling, performing this
step at every measurement would allow particles a great deal of mobility. This mobility is
the enemy of convergence, which is why regularized resampling should only be done when
$\hat{N}_{eff}$ drops very low (say less than 50). Other than the periodic regularized
resampling then, the regularized particle filter is nearly identical to the basic sampling
importance sampling filter (SIS). 

 \begin{algorithmic}
\STATE Initialize $N_p$ Particles: 
        $\{x_i(0),w_i(0) : x_i(0) \sim \alpha(x), w_i(0) = \frac{1}{N_p}, i \in \{1, 2, ... , N_p\} \}$
\STATE $T$ = \{Set of Measurement Times\}
\FOR{$t$ in $T$}
    \FOR{$i$ in $N_p$}
        \STATE $x_i(t) = x_i(t-1) + \int_{t-1}^t f(x(\tau), u(\tau)) d\tau $
        \STATE $w_i(t) = w_i(t-1)\hat{Pr}(y(t) | x(t))$
    \ENDFOR

    \STATE Calculate $N_{eff}$ with \eqref{neff}
    \IF{$N_{eff} < N_R$ (recommend $N_R = min(50, .1N_p)$ )}
        \STATE Calculate empirical $\sigma$ 
        \STATE $h = [N_s8c^{-1}_{n_x}(n_x + 4)(2\sqrt{\pi})^{n_x}]^{\frac{1}{n_x +4}}$
        \STATE Redraw particles using (stratified) basic resampling
        \FOR{$i$ in $N_p$}
            \STATE Draw $\epsilon \sim K$
            \STATE $x_i = x_i + h \sigma \epsilon$
        \ENDFOR
    \ENDIF
\ENDFOR

\STATE At $t + \Delta t$, $t \in T$, $\hat{Pr}(x(t+\Delta t)) \approx 
\sum_{i=1}^{N_p} w_i(t)\delta\left(x - (x_i(t) + \int_t^{t+\Delta t} f(x(\tau), u(\tau)) d\tau) \right)$

 \end{algorithmic}

The ultimate effect of this regularized resampling is a convergence similar to simulated annealing
or a genetic algorithm. Versions of $x$ that are "fit" (give good measurements) spawn more children 
nearby which allow for more accurate estimation near points of high likelihood. 
As the variance of the estimated
$x$'s decrease, the radius in which children are spawned also decreases. Eventually the radius
will approach the width of the underlying uncertainty, $\nu_x$ and $\nu_y$.

\subsection{Choosing $\hat{Pr}(y(T) | x(T))$}
Choosing a representation of an unknown distribution is certainly tricky,
and so the fact that $\hat{Pr}(y(T) | x(T)) = \nu_d + \nu_y$ means that
there is a significant piece of the algorithm that is based primarily conjecture.
Studies of the noise in FMRI typically attribute noise to a Gaussian random
variable or an additive noise process with Gaussian steps. 

\subsubsection{Classical De-trending}
The non-stationary
aspect of a Weiner process as with $\nu_d$ is difficult to compensate for, and so various methods
have been developed to compensate for it. \cite{detrend} and \cite{drift} have
demonstrated that this component is prevalent, and may in fact be a characteristic
of FMRI. In some studies, as many as half the voxels benefit from detrending, meaning
that this is certainly a serious barrier to inference. All the existing methods are performed
during the preprocessing stage, rather than as an integral part of analyzing the BOLD
signal. There is no shortage of theories on the "best" method of detrending, however
a head to head comparison, \cite{detrend}, showed that in most cases subtracting off
a spline works the best. The benefit of the spline versus wavelets, high pass 
filtering or other DC removal techniques is that the frequency response is not set.
A spline is able to move quickly when the signal is moving quickly, and move more
slowly when the signal moves more slowly. That said, the spline will still remove some
amount of signal, just like all of these methods. 

<image of de-spline'd lines with "true" lines>

\subsubsection{Delta Based Inference}
I also propose and test a different method of dealing with the so called "drift". 
Instead of comparing the direct output of the particle filter with the direct
measurement, the algorith compares the change in signal over a single TR. 
In most signal processing cases this would foolish, but that is because the 
general assumption that all noise is high frequency is not the case here.
In fact, every pipeline for the analysis of BOLD signal uses a high pass filter,
but low poss filters are rarely applied, because it is a well known fact that 
most of the signal is in the high frequency range and most of the noise is actually 
in the low frequency range. The particle filter is an 
extremely robust method of inference, and so I would assert that the particle
filter ought to be given as \emph{raw} data as possible. While taking direct measurements
without de-trending would give awful results, using the difference removes the 
DC component and turns a Weiner process into a Gaussian random variable. 

\begin{equation}
\label{measass}
\Delta y = y(t) - y(t-1) = g(x(t)) - g(x(t-1)) + \nu_y(t) - \nu_y(t-1) + \nu_d(t) - \nu(t-1)
\end{equation}

Because $\nu_d$ is a Weiner process, then $\nu_d(t) - \nu_d(t-1)$ is simply a Gaussian step. If
$\nu_d$ is some other additive process, the difference will still be one of a few stable
distributions. If $\nu_y$ is i.i.d. then the resulting distribution will still be zero mean
with a maximum variance of twice the original variance. All the assumptions made originally
for the particle filter hold, and all of the parameters may be distringuished based on
the step sizes, thus it is not unreasonable to attempt to match the string of step sizes
rather than string of direct readings. 

<frequncy response graphs, highlighting noise frequency range and signal frequency range>

\subsubsection{Weighting Function}
Because $\hat{Pr}(y(T) | x(T))$, what I will call the weighting funciton,
is based on an unknown distribution, it is necessary to decide on a function
that will approximate $\hat{Pr}(y(t) | x(T))$. Obviously the function, $\omega(y(t), f(x(t))$
needs to centered at zero and have a scale comparable to the signal levels.
While a Gaussian function is the natural choice, we also wanted to try a distribution
with wider tails, so that outliers don't completely destroy particle's weights.
Therefore, we tried three weighting functions; Gaussian, exponential and 
the Cauchy distribution. The standard deviations (or scale) was set to 
$\sigma_y/5$, where $\sigma_y$ is the variance of all $y[0:\infty]$. Of course 
since the particle filter requires a weighting function to run, this means that
before the particle filter starts all the measurements have to be in. In cases where
this is impossible, a heuristic based on a small sample may work just as well.

The shape of the weighting function is extremely important, because it essentially decides
the rejection rate of particles. A very thin gaussian probability 
distribution function has nice properties, but thin tails. As a result, large
outliers in the measurement vector could easily force all the particles to
have near 0 weights, thus forcing the particle filter to converge improperly. 
On the other end of the spectrum, a cauchy PDF, has relatively fat tails, and
may not weight central particles high enough, preventing the particles from 
converging at a reasonable rate. Exponetials have the benefit of having an 
extremely smooth drop to zero, and a slope of 1 at the origin. Having a non-zero
slope at the origin is beneficial because it discriminates all the way up until
the measured and predicted $y$ are the same. The importance of the weighting function
cannot be overstated, as this is the primary factor in deciding the rate at
which the particle filter converges. 

Obviously the optimal $\hat{Pr}(y(T) | x(T))$ is the true $Pr(y(T) | x(T))$; however
since that is unknown, I tested multiple different distributions. As stated previous,
because the exact scale of $Pr(y(T) | x(T))$ it not even known, the exponential distribution
has the added benefit of having a negative slope all the way form zeo to infinity. Thus
even if the scale is completely wrong, particles will still be well differentiated. 

<Q-Q plot of real fmri data with Gaussian, DC>
<Q-Q plot of real fmri data with Gaussian, deltas>

\section{Methods}
This paper describes two types of experiments; first we will cover
simulations which have the benefit of a ground truth, then we will
cover the methods used in the use of the particle filter on real data.

\subsection{Preprocessing}
As discussed in the section on de-trending, the normal pipeline for analyzing
FMRI involves a great deal of preprocessing. In this paper we make an effort to
minimize any type of preprocessing that will degrade the signal. 
After FMRI data has been acquired it is always necessary to modify the
data in some way to make different runs comparable. Because FMRI signal
levels are unit-less, at the very least it is necessary to convert
the data into \% difference from the baseline. This process removes no data
from signal since it merely subtracting then dividing by a constant. This
is the signal that was input into the delta based particle filter.
Of course there are much more advanced ways of performing this task.
The generally accepted standard is actually to use a high pass filter, although the
cutoff frequency is application dependent and often applied haphazardly.
The high pass filter thus removes the DC component of the signal, and 
some amount of the so called "drift". The problem with this method is that it is
not adaptive to the input. Huge variations in drift frequencies can exist 
in a single time-series. Thus, a single cutoff frequency could miss a significant
drift component, or it could remove \emph{actual} signal, if the cutoff frequency is
set too high. This is why, as I mentioned in the De-trending section, a spline
based detrending method will generally give better results. 

For simulated and real images (tests with multiple time-series), tests were 
also run with and without Gaussian filtering with sigma of %not sure todo
were run, since it is standard
practice to apply a Gaussian spatial filter to the images at each timestep. Obviously
a spatial filter such as Gaussian filtering increased SNR but can also lead to less
precision in the output maps.

\subsection{Simulation}
We performed two types of simulations. First, we simulated a single BOLD time-series based
on a random chosen set of model parameters. This process was relatively straight forward
given the state-space equations for the BOLD signal. After a "true" signal was generated,
we then added a carrier level, since BOLD is typically measured as a \% difference from the
base level. Finally, we added Gaussian noise, and a Weiner Process to the clean signal. The
variance of the Gaussian noise may be expressed in terms of the desired noise SNR, $R$ as:
\begin{equation}
var(y_{noisy}) = var(y) / R
\end{equation}
Since SNR doesn't have quite the same meaning for a Wiener process based noise, the variance 
of the Gaussian steps was set to be:
\begin{equation}
var(y_{noisy}) = var(y) / (4R)
\end{equation}
Once this noisy simulated time series was generated, the exact same particle filter algorithm
that would later be run on full sized images, was run on this single voxel image. We ran
a series of tests to determine the convergence rate of the particle filter, the number
of particles that were required, how weighting functions compared, how different de-trending
methods compared with each other and, finally the variance of the result. By running the exact
time-series with different noise realizations, it was possible to determine the model variance.
As the reader may know, the error of an estimator may be calculated as:
\begin{equation}
MSE(\Theta) = Var(\Theta) + Bias(\Theta)^2
\end{equation}
The variance is an expression of how much the result would change for different noise realizations,
whereas the bias is an expression of how well the model matches the true underlying model. In
this case, because the same model is being used in the particle filter and underlying simulation,
the bias is actually zero. Obviously when this is calculated using \emph{real} data with an unknown
underlying state space equation, there will be some amount of bias error, but assuming that the
noise is similar to the noise used in these tests, the model variance will actually be about the
same. Thus calculating the model variance is extremely helpful in calculating how well determined
our model is, and how consistent it will be for real data. A single timeseries, as opposed to the
thousands present in a real image, makes it easier to 
compare the output with the ground truth, with various parameters. 

Second we used a modified version of the FSL tool 
POSSUM to generate an entire FMRI image from a parameter map. The parameter map was generated
by creating a random image, smoothing it with a large Gaussian kernel, then thresholding
the results. Finally connected regions were each given a set of parameters from a finite
list of randomly chosen parameter sets. The result was a four dimensional (length x width
x height x parameter) image with spatially varying parameters. Time-series of activation level
was generated for each set of parameters, then activation levels were fed into POSSUM's 
function for generating frequency domain data. The patche for POSSUM will be made available.
For each time-series in the simulate FMRI image, the final \emph{static} parameters are saved
into a parameter map. This parameter map may then be compared to the map used to generate the 
simulated data; additionally a new simulation using the calculated parameters may also be 
generated to test the functional difference between the two maps. This would give an absolute 
quantitative difference between the two parameter sets irrespective to parameter slopiness.
So for instance, if $V_0$ is halved, $\epsilon$ doubling may very well give a similar result.
In this case the \% difference between the parameters will be large in each case, but the functional
difference between the parameters will not be great. This is obviously a bad situation, which
is why we wanted to test for it.

\subsection{Real Data}
Finally, we also performed inference based on real FMRI data. The scanner we used...
%... more specifics...

The final result from calculating parameters with the real data was similar to that
from the results from the POSSUM simulated data. The difference being that there was
no ground truth the check it with. 

\section{Results}
\subsection{Single Time-Series Simulation}

Graphs: 

For \{delta, DC/Spline\}, \{exponential, gaussian, cauchy\}, \{biased, unbiased initial\},
\{100, 500, 1000\} particles
\begin{enumerate}
\item Comparison with a linear system with similar number of degrees of freedom
\item Ground truth vs. Estimated signal during particle filter run
\item Ground truth vs. Estimated signal with final parameter set
\item True Parameters vs. Final Parameter Sets
\item Variance of final parameters when faced with same ground truth, different noise
\item Variance of final parameters when faced with same ground truth, different noise
\item MSE of (a new timeseries based on X(t) vs. ground truth) for all t
\item Estimator Variance based on different noise runs
\item Final Particle Distribution
\end{enumerate}

%note to self, epsilon should probably be uniform between 0 and something
\subsection{Simulated Volume}
\begin{enumerate}
\item Parameter Map 
\item Error map of parameters
\item Histogram of \%errors between parameters
\item Activation Map based on a single region with high $\epsilon$, compared with linear
\end{enumerate}

\subsection{FMRI Data}
....

image comparing epsilon-map with GLM activation map

\section{Conclusion}

\bibliographystyle{abbrvnat}
\bibliography{references}


\end{document}

%trash
%stream of new developments in modeling the BOLD signal, nothing has have been
%a plethora of new studies pushing advancements 
%the choice of analysis tools is still relatively limited. Every widely
%available analysis tool is based on parametric modeling, which requires
%prior knowledge of the noise distribution. Indeed, the linear methods
%usually applied are known not to be robust to non-gaussian, non-white
%noise, which is why so many pre-processing steps are necessary.
%These limitations cast a long shadow over any experiment that claims to be
%able to reject the null hypothesis, no matter the P-value. While obviously
%there is a correlation between the BOLD signal and actual activity, its
%possible the current parametric tests are over-estimating the actual amount
%of activity. Its also possible that the current methods underestimate
%actual activity. In this paper we propose a robust model-based approach to 
%activation detection that makes no assumptions about the underlying noise
%model.

