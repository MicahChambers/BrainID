\documentclass[12pt]{article}
\usepackage[left=1in,top=1in,right=1in,nohead]{geometry}

\usepackage[]{times}
\usepackage{verbatim}
\usepackage{amsmath}
\usepackage{amssymb}
\usepackage{multicol}
%\usepackage{setspace}

%\doublespacing

\newcommand{\R}{\mathbb{R}}
\newcommand{\Q}{\mathbb{Q}}
\newcommand{\N}{\mathbb{N}}
\newcommand{\Z}{\mathbb{Z}}
\newcommand{\qed}{\\$\Box$}
\newcommand{\qle}{\stackrel{?}{\le}}
\newcommand{\qeq}{\stackrel{?}{=}}
\newcommand{\closure}{\overline}
\newcommand{\intersect}{\cap}
\newcommand{\union}{\cup}
\newcommand{\nullset}{\emptyset}
\newcommand{\minus}{\ \backslash\ }


\begin{comment}

:Author: Micah Chambers
:Title: Personal Statement

\end{comment}

\author{Micah Chambers}
\title{Research Experience}

%What are all of your applicable experiences?
%For each experience, what were the key questions, methodology, findings, and conclusions?
%Did you work in a team and/or independently?
%How did you assist in the analysis of results?

\begin{document}
\Large
\begin{multicols}{2}
\begin{flushleft}
Micah Chambers
\end{flushleft}

\begin{flushright}
Research Experience
\end{flushright}
\end{multicols}
\normalsize

\subsubsection*{Masters Thesis}
My primary experience in research comes from my masters program at Virginia 
Tech. As a graduate student, beginning in fall 2008, I worked
with Dr. Chris Wyatt on the analysis of FMRI timeseries. The final goal
of the research, which is not yet complete, is to determine the hidden
flow-inducing signal to brain regions, and use
that as input to an Artificial Neural Network (ANN). The reason we are 
performing this step rather than directly using an ANN
is that ANNs of the BOLD signal would
necessarily be more complicated and thus less comparable between
individuals. Because every patient would have a completely
different ANN, it would be difficult to distinguish
pathologies based on network weights.
Additionally, it is important to minimize the number of
parameters when performing
such pattern recognition techniques, to avoid the bias-variance
dilemma. Preprocessing the data to flow-inducing signal thus
removes degrees  of freedom (such as cerbral blood flow, differences in 
blood vessel compliance, etc.) from the neural network.

To determine the underlying
state of the system we have been using particle filters.
Particle filters make no Gaussian assumptions and actually
are able to estimate the noise distribution in parameters. 
The result of a particle filter is joint 
probability distribution function (PDF) of the parameters
defined by a mixture of impulses. 
Thus, if there are several highly likely sets of parameters that could
give equally valid results, all of them are part of the final
distribution.  This is helpful because it gives a far more
complete picture of the system, and could lead to insight
into the source of noise in the system and how parameters interact. 
Additionally having a full probability
distribution for parameters gives a much better impression of
model confidence.

Though the research is not yet complete, the particle filter
has been robust to noise up to an SNR of 2. In fact, we have
been able to correctly estimate 25 model parameters in the face
of such noise. Currently we are working to
validate the particle filter on real data; something 
previous research failed to do. 
We used a basic particle filter library, that I then modified
for our purposes. Most of this project I have done myself,
with guidance from Dr. Wyatt for direction and concepts. 

\subsubsection*{Undergraduate Research}
As an undergraduate student at Virginia Tech, I took part in two
undergraduate research opportunities. 
The first was during the summer after my sophomore year; 
I worked on something called "wireless sensor networks". 
Two of us in the lab worked together for the summer, which 
ended up consisting of a large amounts of troubleshooting
software and not very much "research". I did, however
get exposed to reading papers, and I learned the basics 
of NesC, a C like language used for programming MOTES (the
nodes in wireless sensor networks). 

My second chance at undergraduate research was significantly
better in that I learned a great deal more.
For the first half of the summer, 2007, I worked on the
sonar system for Virginia Tech's Autonomous Underwater vehicle.
Every year in San Diego the  
Association for Unmanned Vehicle Systems International (AUVSI)
has a competition for Autonomous Underwater Vehicles. For
this project I worked alone, and developed dsPIC code to detect
a particular pinger among several based on ping period.
Ultimately the board was able to detect underwater pings of 
particular frequencies, however due to on-board noise, the
hydrophones all detected incoming sound at the same time.
Because of this, our board could detect the pinger, 
but it couldn't properly detect its location. 
I wrote the code to perform beam-forming; however, because 
of the interference beam-forming was impossible. 

Though ultimately the project didn't work, I learned a good deal
about beam-forming, underwater sound, and, naturally, the importance
of testing individual components.

\end{document}
