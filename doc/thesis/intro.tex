\chapter{Introduction}
Traditional methods of analyzing timeseries images produced by 
Function Magnetic Resonance 
Imaging (FMRI) use regression techniques with only linear scaling parameters. 
Though adding more degrees of freedom naturally mandates more computation,
in this thesis I will discuss a Sequential Monte Carlo method of fitting a nonlinear
model with seven degrees of freedom at 
a computation cost that would still allow real time calculations for multiple voxels.
More practically, the method described here forms a completely
separate method of detecting neural activity in comparison to the traditionally
used Statistical Parametric Mapping (SPM). Though more computationally intense,
this method is capable of modeling nonlinear effects, is conceptually simpler
and provides more detailed output. Additionally,
by using a separate particle filter for each voxel's time
series it is possible to estimate parameters and make real-time predictions
for small neural regions, a feature which could be useful towards real time FMRI 
(\cite{DeCharms2005}). Future works will also benefit from the ability to 
apply conditions to the posterior distribution in post-processing without
having to re-run the algorithm; for instance in light of additional physiological
constraints. Modeling the BOLD response as a nonlinear system is the
best way to determine the correlation of stimulus sequence with the BOLD
response; yet in the past doing this on a large scale has been far too
computationally taxing. The solution used here takes approximately 40 seconds
for a single voxel's time series (5 minutes in length, with a quad core machine). 

This thesis is organized as follows. In the introduction I will introduce
FMRI, the method by which neural time changing data is detected. This section
will also describe the basic form of the BOLD model - which drives the 
detectable changes in MR signal. \autoref{sec:Prior Works} will discuss other
methods of analyzing FMRI images as well as other techniques that have
been, or could be applied to the nonlinear regression model described here. 
\autoref{sec:Particle Filter} derives the particle filter using Bayesian 
statistics and discusses some practical elements of implementing the 
particle filter algorithm. \autoref{sec:Methods} then goes into further
detail about the specific particle filter configuration used in this work.
This section also describes the pre-processing used before the particle
filter is applied. The results are described separately for simulated data
and real FMRI data in \autoref{sec:SimulationResults} and \autoref{sec:RealData},
respectively. Finally in \autoref{sec:Conclusion} there is a discussion of
the usefulness and implications of this technique as well as recommendations
for direction of future works. 

\section{Historic Context}
For the past twenty years, Functional Magnetic Resonance Imaging (FMRI) 
has been at the forefront of cognitive research. Despite it's
limited temporal resolution, FMRI is the standard tool for localizing 
neural activation.  Whereas other methods
of analyzing neural signals can be invasive or difficult to acquire, 
FMRI is quick and cheap, and its analysis straight forward.
By modeling the governing equations behind the neural response that
drives FMRI, it is possible to increase the power of FMRI.
The underlying state equations hold important information
about how individual brain regions react to stimuli. The model parameters
on the other hand, hold important information about the patients individual
physiology including existing and future pathologies. In short,
the long chain of events driving FMRI signals contain information 
beyond correlation with stimuli.

In the past fifteen years, a steady stream of studies have built
on the original Blood Oxygen Level Dependent (BOLD) signal 
derivation first described by \cite{Ogawa}.
The seminal work by \cite{Buxton1998} attempted to explain the
time evolution of the BOLD signal using a windkessel model to
describe the local changes in Deoxygenated Hemoglobin content.
Incremental improvements were made to this model until
\cite{Friston2000} brought all the changes together into a single complete 
set of equations. And while there have been numerous adaptations in the model, 
many of them summarized in \cite{Deneux2006}, even the basic versions
have less bias error than the empirically driven \emph{Canonical Hemodynamic Model}
\cite{Deneux2006}, \cite{Handwerker2004}.
On the other hand BOLD signal models have numbers
of parameters ranging from seven \cite{Riera2004} to 50 \cite{Behzadi2005} 
for a signal as short as 100 samples long. This number of parameters presents
a significant risk of being under-determined and having high computation cost. 
In this work, only the simplest physiologically inspired model will be
used (with 7 parameters), and steps will be taken to make the most of computation
time.

\section{Overview}
\label{sec:Introduction Overview}
Detecting neural activity using the changes in FMRI images is based on 
the so called Blood Oxygen Level Dependent (BOLD) signal.
The BOLD signal is caused by minute changes in the ratio of Deoxygenated
Hemoglobin to Oxygenated Hemoglobin in blood vessels throughout the brain.
Because Deoxygenated Hemoglobin (DHb) is paramagnetic, higher concentrations
attenuate the signal detected during T2-weighted Magnetic Resonance Imaging (MRI)
techniques. The most common FMRI imaging technique, due to its rapid repetition 
time (TR), is Echo Planar Imaging (EPI). When axons becomes active,
a large amount of ions quickly flow out of the cell. In order for this
action potential to occur again (and thus for the neuron to fire again),
an active pumping process must move ions back into the
axon. This process of recharging the axon requires extra energy, which temporarily
increases the metabolic rate of oxygen. On a massive scale (cubic millimeter) 
this activation/recharge process happens continuously. However, when a 
particular region of the brain is significantly active, the action potentials
occur more often, resulting in a local increase of the 
Cerebral Metabolic Rate of Oxygen (CMRO2). Thus, blood vessels in an
active area will 
tend to have less oxygenated hemoglobin (due to the increased rate at which
oxygen is being consumed), and more deoxygenated hemoglobin,
resulting in an attenuated FMRI signal. In compensation for 
activation, muscles that
control blood vessels relax in that region to allow more blood flow,
which actually overcompensates.
This ultimately results in lower than average concentration of 
deoxyhemoglobin. Thus, the BOLD signal consists of a short initial
dip in the MR signal, followed by a prolonged increase in signal
that slowly settles out. It is this overcompensation that is the 
primary signal detected with FMRI imaging. This cascade of events
is believed to consist of increased the local metabolism, 
blood flow, blood volume, and oxygenated hemoglobin. The differences
in onsets of these effects is what causes the overcompensation
that is observable in FMRI. Unfortunately, FMRI has no inherent unit
of measurement, and thus signal levels are all relative: within a particular
person, scanner and run. 

\section{FMRI}
Magnetic Resonance Imaging, MRI, is a method of building 3D images
non-invasively, based on the difference between nuclear spin
relaxation times in different molecules. First, the subject 
is brought into a large magnetic field which causes nuclear spins
to align. Radio Frequency (RF) signals may
then be used to excite nuclear spin away from the base alignment. 
As the nuclei precess back to the alignment of the magnetic
field, they emit detectable RF signals. Conveniently, the
excitation of nuclear spins return their original state at different
rates, called the T1 relaxation time, depending on the atoms excited.
Additionally, the
coherence of the spins also decay differently (and roughly an order of 
magnitude faster
than T1 relaxation) based on the properties of the region.
This gives two primary methods of contrasting substances,
which form the basis of T1 and T2 weighted images. Additionally, 
dephasing occurs at two different rates, the T2 relaxation time,
which is unrecoverable, and T2$^*$ relaxation, which is
much faster, but possible to recover from via special RF signals.
T1 relaxation times are typically on the order of seconds if 
a sufficiently strong excitation was applied. 
In order to rapidly acquire entire brain images, as done in Functional 
MRI, a single large excitation pulse is applied to the entire brain,
and the entire volume is acquired in a single T1 relaxation period. 
Because the entire k-space (spatial-frequency) volume is acquired 
from a single excitation, the signal-to-noise-ratio is low
in this type of imaging (Echo Planar Imaging). 

Increasing the spatial resolution of EPI imaging necessarily 
requires more time or faster magnetic field switching. Increasing
magnet switching rates though is difficult, because it can result in
more artifacts, or even lower signal to noise ratios. The result is
that at \emph{best} FMRI is capable of 1 second temporal resolution. 
The signal is further diluted because each voxel contains
the signal from a large number of neurons, capillaries and veins. 
Thus, the FMRI signal, which is sensitive to the chemical composition of 
materials, is the average signal from various types of tissue
in addition to the blood. As mentioned in \autoref{sec:Introduction Overview},
and explored in depth in \autoref{sec:BOLD Physiology},
the usefulness of FMRI comes from the discerning of changes in 
Deoxyhemoglobin/Oxyhemoglobin. Therefore, it is necessary to assume
that in the short term the only chemical changes will be in
capillary beds feeding neurons. In practice this may not be the case, for
instance near significant veins, and it may explain some of the
noise seen in FMRI imaging (see \autoref{sec:Introduction Noise}. 
Because MRI lacks units and certain
areas will have a higher base MR signal, all FMRI studies deal with
percent change from the base signal; rather than raw values. This
also removes most of the structural data which is not helpful 
in determining neural activity.

\section{BOLD Physiology}
\label{sec:BOLD Physiology}
It is well known that the two types of hemoglobin act as a contrast agents in 
EPI imaging
\cite{Buxton1998}, \cite{WEISSKOFF1994}, \cite{Ogawa}, however the connection
between Deoxyhemoglobin/Oxygenated Hemoglobin and neural activity is non-trivial. 
Intuitively, increased 
metabolism will increase Deoxyhemoglobin, however blood vessels are quick
to compensate by increasing local blood flow. Increased inflow, accomplished by loosening 
capillary beds, precedes increased outflow, driving increased 
blood storage capacity.
Since the local MR signal depends on the ratio of Deoxyhemoglobin to Oxygenated
Hemoglobin, increased volume of blood can effect this ratio if 
metabolism doesn't exactly match the increased inflow of oxygenated blood.
This was the impetus
for the ground breaking balloon model (\cite{Buxton1998}) and windkessel
model (\cite{Mandeville1999}). These works derive from first principals
the changes in deoxyhemoglobin ratio and volume of capillaries based on a given flow.
These were the first two attempts to quantitatively account for the shape of the 
BOLD signal as a consequence of the lag between the cerebral blood volume (CBV) 
and the inward cerebral blood flow (CBF). In fact \cite{Buxton1998} went so far as
to show that a simple, well chosen blood flow waveform coupled with a square 
wave cerebral metabolic rate of oxygen (CMRO2) curve, in the context of a balloon 
model, could fully account for the BOLD signal. 

Although \cite{Buxton1998} demonstrated that a well chosen flow waveform could 
explain most features of the BOLD signal, there was still a matter of proposing a
realistic waveform for the CBF and for the CMRO2. \cite{Friston2000} gave
a reasonable and simple
expression for CBF input,$f$, based on a flow inducing signal, $s$, 
in combination with the original balloon model
where $v$ is normalized cerebral blood volume (CBV), $q$ is the normalized
local deoxyhemoglobin/oxygenated hemoglobin ratio.
\begin{eqnarray}
\dot{s} &=& \epsilon u(t) - \frac{s}{\tau_s} - \frac{f - 1}{\tau_f} \\
\dot{f} &=& s\\
\dot{v} &=& \frac{1}{\tau_0}(f - v^\alpha)\\
\dot{q} &=& \frac{1}{\tau_0}(\frac{f(1-(1-E_0)^f)}{E_0} - \frac{q}{v^{1-1/\alpha}})
\label{eq:bold}
\end{eqnarray}
where $\epsilon$ is a neuronal efficiency term, $u(t)$ is a stimulus, and $\tau_f$, $\tau_s$
are both time constants, $E_0$ is the resting metabolic
rate and $\alpha$ is Grubb's parameter controlling the balloon model. 

This completed the basic balloon model, and was well summarized again
in \cite{Riera2003}.  \cite{Obata2004} refined the readout equation 
of the BOLD signal based on the
deoxyhemoglobin content (q) and local blood volume (v), resulting in the
final BOLD equation:
\begin{eqnarray}
y   &=& V_0((k_1 + k_2)(1-q) - (k_2 + k_3)(1-v))\\
k_1 &=& 4.3 \times \nu_0 \times E_0 \times TE = 2.8\\
K_2 &=& \epsilon_0 \times r_0 \times E_0 \times TE = .57\\
k_3 &=& \epsilon_0 - 1 = .43
\label{eq:boldout}
\end{eqnarray}
Where $\nu_0 = 40.3 s^{-1}$  is the frequency offset in Hz for fully
de-oxygenated blood (at 1.5T), $r_0 = 25 s^{-1}$  is the slope relating
change in relaxation rate with change in blood oxygenation, and
$\epsilon_0 = 1.43$ is the 
ratio of signal MR from intravascular to extravascular at rest. Although,
these constants change with experiment ($TE$, $\nu_0$, $r_0$),
patient, and brain 
region ($E_0$, $r_0$), often the estimated values taken from \cite{Obata2004} are 
taken as the constants $a_1 = k_1 + k_2 = 3.4$, and $a_2 = k_2+k_3 = 1$ in 
studies using 1.5 Tesla scanners.
While this model is more accurate than the static Hemodynamic Model used in SPM,
it is not perfect. 

\section{Post Stimulus Undershoot}
\label{sec:Post Stimulus Undershoot}
Although the most widely used, the BOLD model described in \autoref{eq:bold}
and \autoref{eq:boldout} have been extensively added on to. The most
significant feature missing from the original model is the 
post-stimulus undershoot.
The post-stimulus undershoot is the term used for a prolonged subnormal
BOLD response for a period of 10 to 60 seconds after stimulus has
ceased (\cite{Chen2009}, \cite{Mandeville1999a}).

Because \autoref{eq:bold} is not capable of producing such a prolonged undershoot,
additional factors must exist.  Two theories exist for the post stimulus undershoot.
Recall
that a lower than base signal means that there is an increased deoxyhemoglobin
content in the voxel. The first and simplest explanation is that the post-stimulus
undershoot is caused by a prolonged increase in CMRO2 after CBV and CBF
have returned to their base levels. This theory is justified by 
studies that show CBV and CBF returning to the baseline before the BOLD signal
(\cite{Frahm2008}, \cite{Donahue2009}, \cite{Buxton2004}, \cite{Lu2004},
\cite{Shen2008}). Unfortunately, because of limitations on FMRI and in vivo
CBV/CBF measurement techniques it is difficult to isolate whether CBF and
CBV truly have returned to their baseline. Other studies seems to indicate
that there can be a prolonged supernormal CBV (\cite{Mandeville1999a}, 
\cite{Behzadi2005}, \cite{Chen2009a}), although none of these papers completely
rule out the possibility of increased CMRO2. The discrepancies may in part
be explained by a spatial dependence in the post-stimulus undershoot; described
by \cite{Yacoub2006}. \cite{Chen2009}
makes a compelling case that most of the post stimulus undershoot can be 
explained by combination of a prolonged CBV increase, and a prolonged CBF 
undershoot, and that
the previous measurements showing a quick recovery of CBV 
were in fact showing a return to baseline by arterial CBV.

Regardless of the probability that CMRO2 and CBF are detached,
research into the post-stimulus undershoot has led to the creation
of much more in depth models. In \cite{Zheng2002} additional state
variables model oxygen transport, whereas \cite{Buxton2004} models
CMR02 from a higher level, and somewhat more simply; though it 
still adds 9 new parameters. \cite{Behzadi2005}
introduces nonlinearities into the CBF equations as a method to
explain the post-stimulus undershoot, which falls in line with a 
prolonged increase in CBF observed in \cite{Chen2009}. Similarly
\cite{Zheng2005} adds additional compartments to model the BOLD signal
that result from venous and arterial blood. 
\cite{Deneux2006} compared these models and though it did 
not deal extensively with the 
post-stimulus undershoot, it did show incremental improvements
in quality from additional parameters (over the basic Balloon model);
though at the cost of greatly increased complexity.
Importantly,\cite{Deneux2006} did show that by 
simply adding viscoelastic terms from \cite{Buxton2004}, a slowed return 
to baseline is possible to model, without greatly increasing
complexity. Regardless, because these models are more 
complex, and the parameters are not well characterized, in this work the simple
Balloon model is used. Simplicity is even more important because
it is the first time a particle filter has been used to calculate the
BOLD parameters. Of course, future works could certainly benefit from
the more advanced models, especially with the addition of viscoelastic
effects.

In summary, there have been extensive refinements to the Balloon
model,however, the increased complexity and lack of known priors 
make these models less desirable. Therefor, in this work
work, where computation time is especially important, only the 
most basic model is used.

\section{Properties of the BOLD Model}
\label{sec:BOLD Analysis}
Since the first complete BOLD model was proposed by \cite{Friston2002}, 
several studies have analyzed its properties. 
The most important property is that the system is dissipative, and given
enough time will converge to a constant value. This is found simply by
analyzing the eigenvalues of the Jacobian of the state equations, 
(\cite{Deneux2006}, \cite{Hu2009}). The steady state of the Balloon
model equations gives:

\begin{eqnarray}
s_{ss} &=& 0 \nonumber \\
f_{ss} &=& \tau_f\epsilon u + 1\nonumber \\
v_{ss} &=& (\tau_f\epsilon u + 1)^\alpha\nonumber \\
q_{ss} &=& \frac{(\tau_f\epsilon u + 1)^\alpha}{E_0}(1-(1-E_0)^{1/(\tau_f\epsilon u + 1)})\nonumber \\
y_{ss} &=& V_0((k_1+k_2)(1-q_{ss}) - (k_2+k_3)(1-v_{ss}))
\label{eq:steadystate}
\end{eqnarray}

In real FMRI data, there is a significant nonlinearity in response; with short sequences
responding disproportionately strong (\cite{Birn2001}, \cite{Wager2005}, \cite{Deneux2006}).
This nonlinearity is accounted for in the Balloon model, although \cite{Deneux2006}
shows that if there will be large variance in the length of signals, 
modeling Neural Habituation may be necessary to fully capture the range of responses. 
Stimuli that last longer than 4 seconds 
tend to be more linear, which is why block designs are so well accounted for
by the General Linear Model (\cite{Birn2001}, \cite{Deneux2006}). For this 
paper we will use only a simple version of the Balloon model, described by
\autoref{eq:bold}, to keep the solution tractable, and because it is the most
well studied. 

Another interesting result of \cite{Deneux2006} was the sensitivity analysis.
There it was found that the parameters are far from perpendicular,
implying that exact inference of parameters may not be possible without
constraint. This could explain the extreme discrepencies in \autoref{tab:Params}.

A summary of results from previous studies of the Balloon Parameters
are shown in \autoref{tab:Params}.

%note to self, friston2002b's parameters are from a picture

\begin{table}[t]
\centering
\begin{tabular}{|c || c | c | c | c|}
\hline 
Parameter  & \cite{Friston2000} & \cite{Johnston2008} & \cite{Vakorin2007} & \cite{Deneux2006}\\
\hline
$\tau_0  $ &  $N(.98 , .25^2)$  & $8.38 \pm 1.5  $ & $.94$ & .27\\
$\alpha  $ &  $N(.33 , .45^2)$  & $.189 \pm .004 $ & $.4$ (NC) & .63 \\
$E_0     $ &  $N(.34 , .1 ^2)$  & $.635 \pm .072 $ & $.6$ (NC) & .33\\
$V_0     $ &  $.03$ (NC)        & $.0149 \pm .006$ & (NC) & .16\\
$\tau_s  $ &  $N(1.54, .25^2)$  & $4.98 \pm 1.07 $ & $2.2$ & 2.04 \\
$\tau_f  $ &  $N(2.46, .25^2)$  & $8.31 \pm 1.51 $ & $.45$ & 5.26\\
$\epsilon$ &  $N(.54 , .1 ^2)$  & $.069 \pm .014 $ & (NC) & .89\\
\hline
\end{tabular}
\caption{Parameters found by various studies. (NC) indicates that the value
wasn't calculated. \cite{Vakorin2007} made use of the values from \cite{Friston2002}
where not explicitly stated}
\label{tab:Params} 
\end{table}

